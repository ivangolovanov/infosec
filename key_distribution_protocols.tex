\chapter{Распространение ключей}\index{протокол!распространения ключей}\label{chapter-key-distribution-protocols}
\selectlanguage{russian}

Задачей распространения ключей между двумя пользователями является создание секретных псевдослучайных сеансовых ключей шифрования и аутентификация сообщений. Пользователи предварительно создают и обмениваются ключами аутентификации один раз. В дальнейшем для создания защищённой связи пользователи производят взаимную аутентификацию и вырабатывают сеансовые ключи\index{ключ!сеансовый}.

\section[Трёхэтапный протокол Шамира]{Трёхэтапный протокол Шамира на коммутативных шифрах}
\selectlanguage{russian}

Предположим, что две стороны $A$ и $B$ соединены незащищённым каналом связи. Каждая из этих сторон имеет свой секретный ключ: $A$ имеет ключ $K_A$, $B$ имеет ключ $K_B$. Сторона $A$ должна создать общий секретный ключ $K$ и передать стороне $B$.

Для решения этой задачи используют трёхэтапный протокол Шамира с тремя <<замками>>. \emph{Протокол Шамира}\index{протокол!Шамира} построен на \emph{коммутативных} функциях шифрования, для которых выполняется условие:
    \[ E_{K_{B}} (E_{K_{A}}(K))=E_{K_{A}} (E_{K_{B}}(K)). \]

Протокол предполагает следующие процедуры:
\begin{enumerate}
    \item $A$ создаёт секретный ключ $K$, шифрует его своей системой шифрования с помощью своего ключа $K_A$ и посылает сообщение стороне $B$:
        \[ A \rightarrow B: ~ E_{K_A}(K). \]
    \item $B$ получает это сообщение, шифрует его с помощью своего ключа $K_B$ и посылает сообщение стороне $A$:
        \[ A \leftarrow B: ~ E_{K_B}( E_{K_A}( K)). \]
    \item Сторона $A$, получив сообщение $E_{K_B}(E_{K_A}(K))$, использует свой секретный ключ $K_A$ для расшифрования:
            \[ D_{K_A}(E_{K_B} (E_{K_A}(K))) = E_{K_B}(K). \]
        Сторона $A$ передаёт стороне $B$ сообщение:
        \[ A \rightarrow B: ~ E_{K_B}(K). \]
    \item Сторона $B$, получив сообщение $E_{K_B}(K)$, использует свой секретный ключ $K_B$ для расшифрования:
            \[ D_{K_B}(E_{K_B}(K)) = K. \]
        В результате стороны получают общий секретный ключ $K$.
\end{enumerate}

Приведём пример неудачного шифрования с использованием коммутативных функций.

\begin{enumerate}
    \item $A$ имеет функцию шифрования совершенной секретности $E_{K_A}(K) = K \oplus K_A$, где $K_A$ -- двоичная последовательность с равномерным распределением символов. $A$ посылает это сообщение стороне $B$:
            \[ A \rightarrow B: ~ E_{K_A}(K) = K \oplus K_A. \]
    \item $B$ использует такую же функцию шифрования совершенной секретности с ключом $K_B$ (двоичная последовательность с равномерным распределением символов). $B$ шифрует полученное сообщение и отправляет $A$:
            \[ A \leftarrow B: ~ E_{K_A}(K) \oplus K_B = K \oplus K_A \oplus K_B. \]
    \item Сторона $A$, получив сообщение $K \oplus K_A \oplus K_B$, выполняет расшифрование:
            \[ K \oplus K_A \oplus K_B \oplus K_A = K \oplus K_B. \]
        Сторона $A$ передаёт стороне $B$ сообщение:
            \[ A \rightarrow B: ~ K \oplus K_B. \]
    \item Сторона $B$, получив сообщение $K \oplus K_B$, выполняет расшифрование:
            \[ K \oplus K_B \oplus K_B = K. \]
        Обе стороны получают общий секретный ключ $K$.
\end{enumerate}

Предложенный выбор коммутативной функции шифрования совершенной секретности был назван неудачным, так как существуют ситуации, при которых криптоаналитик может определить ключ $K$. Предположим, что криптоаналитик перехватил все три сообщения:
    \[ K \oplus K_A, ~~ K \oplus K_A \oplus K_B, ~~ K \oplus K_B. \]
Сложение по модулю 2 всех трёх сообщений даёт ключ $K$. Поэтому такая система шифрования не применяется.

Теперь приведём протокол надёжной передачи секретного ключа, основанный на экспоненциальной (коммутативной) функции шифрования. Стойкость этого протокола связана с трудностью задачи вычисления дискретного логарифма: известны значения $y, g, p$, найти $x$ в уравнении $y = g^x \mod p$.

\textbf{Протокол Шамира распространения ключей}

Выбирают большое простое\index{число!простое} число $p\sim 2^{1024}$ и используют его как открытый ключ.

\begin{enumerate}
    \item Сторона $A$ задаёт общий секретный ключ $K <p$ и выбирает целое число $a$, взаимно простое с $p-1$. $A$ вычисляет и посылает сообщение стороне $B$:
            \[ A \rightarrow B: ~ K^a \mod p. \]
        Существует число $c$ такое, что $a c =1 \mod (p-1)$, то есть $a c = 1 + l (p-1)$, где $l$ -- целое число. Число $c$ будет использовано стороной $A$ на следующем этапе.
    \item Сторона $B$ выбирает целое число $b$, взаимно простое с $p-1$. Используя полученное сообщение, вычисляет и посылает сообщение стороне $A$:
            \[ A \leftarrow B: ~ (K^a)^b \mod p. \]
    \item Сторона $A$, получив сообщение, вычисляет
        \[ \left( K^{ab} \right)^c = K^{(1 + l (p-1)) b} = K^b \cdot K^{l (p-1) b} = K^b \mod p. \]
        Здесь применена малая теорема Ферма\index{теорема!Ферма малая}: $K^{p-1} = 1 \mod p$, поэтому $\left( K^{p-1} \right)^{lb} = 1 \mod p$.
        $A$ посылает $B$ сообщение:
            \[ A \rightarrow B: ~ K^b \mod p. \]
    \item Сторона $B$, получив сообщение $K^{b}\mod p$, вычисляет
        \[ (K^b \mod p)^d = K^{bd} \mod p = K, \]
        где $d$ найдено из $b d =1 \mod (p-1)$.
\end{enumerate}

Теперь проверим криптостойкость этого протокола. Предположим, что криптоаналитик перехватил три сообщения:
\[ \begin{array}{l}
    y_1 = K^a \mod p, \\
    y_2 = K^{ab} \mod p, \\
    y_3 = K^b \mod p. \\
\end{array} \]
Чтобы найти ключ $K$, надо решить систему из этих трёх уравнений, что имеет очень большую вычислительную сложность, неприемлемую с практической точки зрения, если все три числа $a, b, ab$ достаточно велики. Предположим, что $a$ (или $b$) мало. Тогда, вычисляя последовательные степени $y_3$ (или $y_1$), можно найти $a$ (или $b$), сравнивая результат с $y_2$. Зная $a$, легко найти $a^{-1}\mod(p-1)$ и $K=(y_1)^{a^{-1}}\mod p$.

Недостатком этого протокола является отсутствие аутентификации сторон. Следовательно, нужно дополнительно использовать цифровую подпись при передаче сообщения.


\section{Симметричные протоколы}

\subsection{Аутентификация и атаки воспроизведения}

Рассмотрим такую ситуацию: обе стороны $A$ и $B$ имеют общий долговременный ключ $K_{AB}$ и симметричную систему шифрования. Нужно выработать сеансовый секретный ключ $K$. Сторона $A$ создаёт ключ $K$ и желает его передать стороне $B$.

\begin{enumerate}
    \item Для этого сторона $A$ с помощью общего ключа $K_{AB}$ создаёт и передаёт $B$ шифрованное сообщение:
            \[ A \rightarrow B: ~ E_{K_{AB}}(K, B, A). \]
        В этом сообщении имеются так называемые поля -- $(B,A)$ -- информация для дополнительного подтверждения.
    \item Сторона $B$, используя общий ключ $K_{AB}$, расшифровывает полученное сообщение:
            \[ D_{K_{AB}}( E_{K_{AB}}( K, B, A)) = (K, B, A). \]
        В результате сторона $B$ получает сеансовый ключ $K$ и дополнительные данные $(B,A)$.
\end{enumerate}

Недостаток этого протокола состоит в том, что криптоаналитик может перехватывать сообщения и через некоторое время переслать их стороне $A$.

Рассмотрим другие варианты решения задачи о передаче сеансового ключа.
Задача остаётся прежней: обе стороны $A$ и $B$ имеют общий долговременный секретный ключ $K_{AB}$, сторона $A$ должна выработать сеансовый секретный ключ $K$ и доставить его стороне $B$.

Протокол включает \emph{метки времени} -- информацию о моменте $t_A$ отправки сообщения и моменте получения сообщения $t_B$.

\begin{enumerate}
    \item Сторона $A$ вырабатывает $K$, с помощью долговременного ключа $K_{AB}$ создаёт шифрованное сообщение с меткой времени $t_A$ и передаёт его стороне $B$:
            \[ A \rightarrow B: ~ E_{K_{AB}}(K, t_A). \]
    \item Сторона $B$ получает сообщение и расшифровывает его с помощью общего ключа:
            \[ D_{K_{AB}}( E_{K_{AB}}( K, t_A)) = (K, t_A). \]
        В результате $B$ получает $(K, t_A)$, то есть секретный ключ и метку времени. $B$ измеряет время прихода $t_B$ и интервал запаздывания. Если $|t_B - t_A| \leq \delta$, то $B$ аутентифицирует $A$.
\end{enumerate}
Метка времени является одноразовой меткой и защищает от атак воспроизведения ранее записанных сообщений.

Рассмотрим другой способ передачи ключа с дополнительной информацией в виде \emph{одноразовых случайных меток} (nonce -- number used once) вместо меток времени. Протокол передачи состоит в следующем.

\begin{enumerate}
    \item Сторона $A$ вырабатывает случайное число $r_A$, шифрует сообщение, в котором $(r_A, A)$ -- реквизиты $A$, и передаёт его стороне $B$:
            \[ A \rightarrow B: ~ E_{K_{AB}}(r_A, A). \]
    \item Сторона $B$ вырабатывает сеансовый ключ $K$, создаёт шифрованное сообщение и посылает его $A$:
            \[ A \leftarrow B: ~ E_{K_{AB}}(K, r_A, A). \]
    \item Сторона $A$ расшифровывает полученное сообщение:
            \[ D_{K_{AB}}( E_{K_{AB}}( K, r_A, A)) = (K, r_A, A). \]
        В результате $A$ получает сеансовый ключ и подтверждение своих реквизитов, что является дополнительной аутентификацией.
\end{enumerate}

Предположим, что сторона $B$ тоже желает убедиться, что имеет дело со стороной $A$. Тогда этот протокол следует дополнить передачей реквизитов $B$. По-прежнему считаем, что у $A$ и $B$ общая система шифрования с долговременным секретным ключом $K_{AB}$.

\begin{enumerate}
    \item Сторона $A$ вырабатывает случайное число $r_A$, шифрует и передаёт стороне $B$ сообщение, в котором $(r_A, A)$ -- реквизиты $A$:
            \[ A \rightarrow B: ~ E_{K_{AB}}(r_A, A). \]
    \item Сторона $B$ вырабатывает случайное число $r_B$ и отправляет стороне $A$ зашифрованное сообщение:
            \[ A \leftarrow B: ~ E_{K_{AB}}(K_B, r_B, r_A, A), \]
        где $K_B$ -- ключ $B$.
     \item Сторона $A$ осуществляет расшифрование:
            \[ D_{K_{AB}}(E_{K_{AB}}(K_B, r_B, r_A, A)) = (K_B, r_B, r_A, A), \]
        и получает ключ $K_B$ и реквизиты $r_B, r_A, A$. Для аутентификации себя сторона $A$ создаёт свой ключ $K_A$ и отправляет стороне $B$ шифрованное сообщение:
            \[ A \rightarrow B: ~ E_{K_{AB}}(K_A, r_B, r_A, B). \]
     \item Сторона $B$ осуществляет расшифрование:
            \[ D_{K_{AB}}(E_{K_{AB}}(K_A, r_B, r_A, B)) = (K_A, r_B, r_A, B), \]
        которое определяет ключ $K_A$ и аутентифицирует $A$.
\end{enumerate}

Таким образом, обе стороны имеют в своём распоряжении ключи $K_A, K_B$ в качестве сеансовых секретных ключей.

\subsection{Протокол с ключевым кодом аутентификации}

При использовании хэш-функции $K = h(K_{A} ~\|~ K_{B})$ происходит усиление секретности. Здесь $(K_{A} ~\|~ K_{B})$ -- конкатенация $K_{A} $ и $K_{B}$.

% Достоинства: предположим, $K_{A} ,K_{B} $ -- не обладают «хорошими» свойствами случайности (биты распределены неравномерно или зависимы друг от друга), то есть, $P_{K_{A} ,K_{B} } (0)=\frac{1}{2} -\varepsilon $, где $\varepsilon $ - мало, но не 0. Тогда вероятность того, что этот бит в \emph{K }будет равным нулю, $P_{K} (0)=\frac{1}{2} -\varepsilon ',\varepsilon '<\varepsilon $- усиление секретности.

Вычисление хэш-значения, как правило, выполняется быстрее, чем расшифрование. Поэтому были разработаны протоколы, в которых вместо функции шифрования используется имитовставка\index{имитовставка} на основе хэш-функции ($\MAC_K$). Рассмотрим протокол такого рода.
\begin{enumerate}
    \item Сторона $A$ вырабатывает сеансовый ключ $K$, использует одноразовую метку $t_{A}$, создаёт и пересылает стороне $B$ сообщение:
            \[ A \rightarrow B: ~ t_A, ~ B, ~ K \oplus \MAC_{K_{AB}}( t_A, B), ~ \MAC_{K_{AB}}(K, t_A, B). \]
    \item Сторона $B$ вычисляет
            \[ \MAC_{K_{AB}}(t_A, B) \oplus K \oplus \MAC_{K_{AB}}(t_A, B) = K \]
        и получает сеансовый ключ $K$.
\end{enumerate}

Заметим, что криптоаналитик может добавить в поле случайную последовательность, тогда вместо $K$ получаем <<$K$ плюс помеха>>. Вмешательство криптоаналитика может быть выявлено благодаря наличию четвёртого поля в сообщении. Используя полученное значение $K$, вычисляют $\MAC_{K_{AB}}(K, t_A, B)$ и сравнивают с четвёртым полем. Если совпадает, то вмешательства криптоаналитика не было.

\subsection{Протокол Нидхема~---~Шрёдера с доверенным центром}\index{протокол!Нидхема~---~Шрёдера}
\selectlanguage{russian}

Рассмотрим ситуацию, когда в сети имеется некоторый надёжный (доверенный) сервер (центр) $T$, которому доверяют все пользователи сети. Сервер для работы с абонентами сети использует некоторую систему шифрования $E_S(*)$, где ключ $S=K_{AT}$ известен только $A$ и $T$, и неизвестен остальным участникам сети, $S = K_{BT}$ известен только $B$ и $T$. Предполагаем, что сервер имеет хороший генератор случайных чисел. Сеансовый ключ сервер вырабатывает по запросу. Стороны $A$ и $B$ могут выбирать разные одноразовые метки.

Приведём в качестве примера упрощённую версию известного \emph{протокола Нидхема~---~Шрёдера} (Needham~---~Schroeder) с симметричным шифром.
\begin{enumerate}
    \item Сторона $A$ передаёт серверу $T$ реквизиты сторон $A$ и $B$ и некую одноразовую метку $N_A$, которая может быть, например, меткой времени или случайным (одноразовым) числом, что оговаривается заранее:
            \[ A \rightarrow T: ~ A, B, N_A. \]
    \item Сервер $T$ вырабатывает секретный сеансовый ключ $K$ для $A$ и $B$ и отправляет стороне $A$ шифрованное сообщение:
            \[ A \leftarrow T: ~ E_{K_{AT}}(N_A, B, K, E_{K_{BT}}(K, A)). \]
    \item Сторона $A$ расшифровывает сообщение:
            \[ D_{K_{AT}}( E_{K_{AT}}(N_A, B, K, E_{K_{BT}}(K, A))) = (N_A, B, K, E_{K_{BT}}(K, A)) \]
        и, чтобы доставить ключ, передаёт стороне $B$ сообщение:
            \[ A \rightarrow B: ~ E_{K_{BT}}(K, A). \]
    \item Сторона $B$ расшифровывает полученное сообщение:
            \[ D_{K_{BT}}( E_{K_{BT}}( K,A)) = (K,A) \]
        и получает ключ и реквизиты $A$, которые требуются для того, чтобы сторона $B$ знала, кому отвечать. Кроме того, сторона $B$ дополнительно желает идентифицировать сторону $A$. Для этого $B$ пересылает $A$ зашифрованную одноразовую метку:
            \[ A \leftarrow B: ~ E_{K}(N_B). \]
    \item Сторона $A$ расшифровывает:
            \[ D_K( E_K( N_B)) = N_B \]
        и возвращает $B$ изменённую одноразовую метку:
            \[ A \rightarrow B: ~ E_K(N_B + 1). \]
    \item Сторона $B$ расшифровывает:
            \[ D_K( E_K( N_B + 1)) = N_B + 1, \]
        проверяет $N_B$ и убеждается, что имеет дело со стороной $A$.
    \item Если требуется двусторонняя аутентификация, то аналогично поступают со стороной $A$: на некотором этапе вносится одноразовая метка $N_A$.
\end{enumerate}


\section{Асимметричные протоколы}

Асимметричные протоколы, или же протоколы, основанные на криптосистемах с открытыми ключами, позволяют ослабить требования к предварительному этапу протоколов. Вместо общего секретного ключа, который должны иметь две стороны (либо обе стороны и доверенный центр), в рассматриваемых ниже протоколах стороны должны предварительно обменяться открытыми ключами (между собой либо между собой и доверенным центром). Такой предварительный обмен может проходить по открытому каналу связи, в предположении, что криптоаналитик не может повлиять на содержимое канала связи на данном этапе.

\subsection{Простой протокол}

Рассмотрим протокол распространения ключей с помощью асимметричных шифров. Введём обозначения: $K_B$ -- открытый ключ стороны $B$, а $K_A$ -- открытый ключ стороны $A$. Протокол включает три сеанса обмена информацией.
\begin{enumerate}
    \item В первом сеансе сторона $A$ посылает стороне $B$ сообщение:
            \[ A \rightarrow B: ~ E_{K_B}(K_1, A), \]
        где $K_1$ -- ключ, выработанный стороной $A$.
    \item Сторона $B$ получает $(K_1, A)$ и передаёт стороне $A$ наряду с другой информацией свой ключ $K_2$ в сообщении, зашифрованном с помощью открытого ключа $K_A$:
            \[ A \leftarrow B: ~ E_{K_A}(K_2, K_1, B). \]
    \item Сторона $A$ получает и расшифровывает сообщение $(K_2, K_1, B)$. Во время третьего сеанса сторона $A$, чтобы подтвердить, что она знает ключ $K_2$, посылает стороне $B$ сообщение:
            \[ A \rightarrow B: ~ E_{K_B}(K_2). \]
\end{enumerate}
Общий ключ формируется из двух ключей: $K_1$ и $K_2$.

\subsection{Протоколы с цифровыми подписями}

Существуют протоколы обмена, в которых перед началом обмена ключами генерируются подписи сторон $A$ и $B$, соответственно $S_A(m)$ и $S_B(m)$. В этих протоколах можно использовать различные одноразовые метки. Рассмотрим пример.
\begin{enumerate}
    \item Сторона $A$ выбирает ключ $K$ и вырабатывает сообщение:
            \[ \left( K, ~ t_A, ~ S_A(K, t_A, B) \right), \]
        где $t_A$ -- метка времени. Зашифрованное сообщение передаёт стороне $B$:
        \[ A \rightarrow B: ~ E_{K_B}(K, ~ t_A, ~ S_A(K, t_A, B)). \]
    \item Сторона $B$ получает это сообщение, расшифровывает $\left( K, ~ t_A, ~ S_A(K, t_A, B) \right)$ и вырабатывает свою метку времени $t_B$. Проверка считается успешной, если $|t_B - t_A | < \delta $. Сторона $B$ знает свои реквизиты и может осуществлять проверку подписи.
\end{enumerate}

Имеется второй вариант протокола, в котором шифрование и подпись выполняются раздельно.
\begin{enumerate}
    \item Сторона $A$ вырабатывает ключ $K$, использует одноразовую метку (или метку времени) $t_{A}$ и передаёт стороне $B$ два различных зашифрованных сообщения:
            \[ \begin{array}{ll}
                A \rightarrow B: & ~ E_{K_B}(K, t_A), \\
                A \rightarrow B: & ~ S_A(K, t_A, B). \\
            \end{array} \]
    \item Сторона $B$ получает это сообщение, расшифровывает $K, t_A$ и, добавив свои реквизиты $B$, может проверить подпись $S_A(K, t_A, B)$.
\end{enumerate}

В третьем варианте протокола сначала производится шифрование, потом подпись.
\begin{enumerate}
    \item Сторона $A$ вырабатывает ключ $K$, использует одноразовую случайную метку или метку времени $t_A$ и передаёт стороне $B$ сообщение:
        \[ A \rightarrow B: ~ t_A, ~ E_{K_B}(K, A), ~ S_A(t_A, ~ K, ~ E_{K_B}(K, A)). \]
    \item Сторона $B$ получает это сообщение, расшифровывает $\left( t_A, ~ K, ~ A, ~ E_{K_B}(K, A) \right)$ и проверяет подпись $S_A(t_A, ~ K, ~ E_{K_B}(K, A))$.
\end{enumerate}

\input{diffie-hellman}

%\section{Протоколы с аутентификацией}

\subsection{Односторонняя аутентификация}

\input{el-gamal_protocol}

\input{mti}

\input{sts}

\input{girault_scheme}

\input{bloms_scheme}

В этом разделе были рассмотрены протоколы, в которых ключи вырабатываются в процессе обмена информацией.
%Существует и другой подход, который будет рассмотрен в следующих разделах.
