\documentclass[10pt,a4paper,openany]{book}
%\documentclass[12pt,report,russian]{ncc}
%\usepackage{a4wide}
% Для векторых русских шрифтов в PDF не забудьте установить пакеты cm-super & cm-unicode
\usepackage{cmap}                       % Поддержка поиска русских слов в PDF (pdflatex)
\usepackage[X2, T2A]{fontenc}
%\usepackage[T2, OT1]{fontenc}
\usepackage[utf8]{inputenc}
\usepackage[english,french,german,italian,latin,russian]{babel}
\usepackage{indentfirst}                % Красная строка в первом абзаце
%\usepackage{misccorr}
%Может быть установлено 8pt, 9pt, 10pt, 11pt, 12pt, 14pt, 17pt, and 20pt
%\usepackage[12pt]{extsizes}
%\usepackage[mag=1000,a4paper,left=3cm,right=2cm,top=2cm,bottom=2cm,noheadfoot]{geometry}

% Подключение amsmath также даёт поддержку автоматических \dots
% см. http://tex.stackexchange.com/questions/77737/dots-versus-ldots-is-there-a-difference
% см. http://tex.stackexchange.com/questions/117730/what-is-the-difference-between-ldots-and-cdots
\usepackage{amsmath} % разрешить \texttt и аналогичные в формулах
\usepackage{amssymb} % дополнительные математические символы
\usepackage{graphicx} % поддержка изображений

%\usepackage{amsfonts, eucal, bm, color, }

\usepackage{algorithm, algorithmic}     % 'algorithm' environments
\floatname{algorithm}{Алгоритм}

\usepackage{arydshln}                   % dash lines in tables
\usepackage{caption}                    % titles for figures
\usepackage{csquotes}
\usepackage{enumerate}
\usepackage{enumitem}                   % кастомизация itemize/enumerate, напр. отказ от indent
\usepackage{fancybox}                   % страница в рамке
\usepackage{float}			% sub figures
\usepackage[totoc=true]{idxlayout}      % балансировка индексов на последней странице, индекс в ToC
\usepackage{lscape}                     % поддержка поворота страниц на 90 градусов для широких таблиц
\usepackage{makeidx}                    % index
\usepackage{multicol}                   % поддержка колонок
\setlength{\columnsep}{0.1cm}
\usepackage{multirow}                   % multirow cells in tables
%\usepackage{subfig}			% sub figures
\usepackage{subcaption}         % sub figures, incompatible with package{subfig}
\usepackage{tikz}                       % векторная графика внутри TeX
\usepackage{tablefootnote}		% footnote в таблицах
\usepackage{wrapfig}			% sub figures
\usepackage{textcomp}                   % \No support

\usepackage[left=1.84cm, right=1.5cm, paperwidth=14cm, top=1.8cm, bottom=2cm, height=19.8cm, paperheight=20cm]{geometry}
\usepackage[parentracker=true,
  backend=biber,
  hyperref=auto,
  language=auto,
  citestyle=gost-numeric,
  defernumbers=true,
  bibstyle=gost-numeric,
  sortlocale=ru_RU
]{biblatex}								% библиография по ГОСТу
\addbibresource{bibliography.bib}

% поддержка гиперссылок; гиперссылки в pdf, должен быть последним загруженным пакетом
\ifx\pdfoutput\undefined
    \usepackage[unicode,dvips]{hyperref}
\else
    \usepackage[pdftex,colorlinks,unicode,bookmarks]{hyperref}
\fi

%\paperwidth=16.8cm \oddsidemargin=0cm \evensidemargin=0cm \hoffset=-0.33cm \textwidth=13.2cm
%\paperheight=24cm \voffset=-0.4cm \topmargin=0cm \headsep=0cm \headheight=0cm \textheight=19.8cm \footskip=0.9cm

% параметры PDF файла
\hypersetup{
    pdftitle={Защита информации},
    pdfauthor={Э. М. Габидулин, А. С. Кшевецкий, А. И. Колыбельников, С. М. Владимиров},
    pdfsubject=учебное пособие,
    pdfkeywords={защита информации, криптография, МФТИ}
}

% добавить точку после номера секции, раздела и~т.\,д.
\makeatletter
\def\@seccntformat#1{\csname the#1\endcsname.\quad}
\def\numberline#1{\hb@xt@\@tempdima{#1\if&#1&\else.\fi\hfil}}
\makeatother

% перенос слов с тире
%\lccode`\-=`\-
%\defaulthyphenchar=127

% изменить подписи к рисункам, таблицам и~т.\,д.
\captionsetup{labelsep=endash}          % Нумерационный заголовок и текст разделяются тире
\captionsetup{textformat=simple}        % Текст подписи будет напечатан как есть
%\captionsetup[table]{position=above}    % вертикальные отступы подписи таблицы для случая, когда подпись вверху
%\captionsetup[figure]{position=below}   % вертикальные отступы подписи рисунка для случая, когда подпись внизу

%% стиль главы и секции вверху страницы
%\pagestyle{fancy}
%%\renewcommand{\chaptermark}[1]{\markboth{#1}{}}
%\renewcommand{\sectionmark}[1]{\markright{#1}{}}
%
%%\fancyhf{}
%%\fancyfoot[СE,CO]{\thepage}
%%\fancyhead[LE]{\textsc{\nouppercase{\leftmark}}}
%\fancyhead[RO]{\textsc{\nouppercase{\rightmark}}}
%
%\fancypagestyle{plain}{ %
%\fancyhf{}                              % remove everything
%\renewcommand{\headrulewidth}{0pt}      % remove lines as well
%\renewcommand{\footrulewidth}{0pt}}

% запретить выходить за границы страницы
\sloppy

\newtheorem{theorem}{Теорема}[section]
\newtheorem{lemma}[theorem]{Лемма}
\newtheorem{definition}[theorem]{Определение}
\newtheorem{property}[theorem]{Утверждение}
\newtheorem{corollary}[theorem]{Следствие}
%\newtheorem{algorithm}[theorem]{Алгоритм}
\newtheorem{remark}[theorem]{Замечание}
\newcommand{\proof}{\noindent\textsc{Доказательство.\ }}

%\newtheorem{example}{\textsc{\textbf{Пример}}}
\newcommand{\example}{\textsc{\textbf{Пример.}} }
\newcommand{\exampleend}

\DeclareMathOperator{\ord}{ord}
\newcommand{\set}[1]{\mathbb{#1}}
\newcommand{\group}[1]{\mathbb{#1}}
\newcommand{\E}{\group{E}}
\newcommand{\F}{\group{F}}
\newcommand{\GF}[1]{\group{GF}(#1)}
\newcommand{\Gr}{\group{G}}
\newcommand{\Mod}{\operatorname{mod}}
\newcommand{\R}{\group{R}}
\newcommand{\Z}{\group{Z}}
\newcommand{\MAC}{\textrm{MAC}}
\newcommand{\HMAC}{\textrm{HMAC}}
\newcommand{\PK}{\textrm{PK}}
\newcommand{\SK}{\textrm{SK}}

\newcommand{\langde}[1]{нем. \foreignlanguage{german}{\textit{#1}}}
\newcommand{\langfr}[1]{фр. \foreignlanguage{french}{\textit{#1}}}
\newcommand{\langen}[1]{англ. \foreignlanguage{english}{\textit{#1}}}
\newcommand{\langit}[1]{итал. \foreignlanguage{italian}{\textit{#1}}}
\newcommand{\langlat}[1]{лат. \foreignlanguage{latin}{\textit{#1}}}

% Русская типографика
\renewcommand\leq{\leqslant}
\renewcommand\geq{\geqslant}
\renewcommand\emptyset{\varnothing}
\renewcommand\kappa{\varkappa}
\renewcommand\epsilon{\varepsilon}
\renewcommand\phi{\varphi}

% Для раздела с задачами
\newcommand{\taskinit}{\newcounter{task-section}\setcounter{task-section}{0}\newcounter{task-number}}
\newcommand{\tasksection}{\addtocounter{task-section}{1}\setcounter{task-number}{0}}
\newcommand{\tasknumber}{\textbf{\No\addtocounter{task-number}{1}\arabic{task-section}.\arabic{task-number}.}~~}

%Наконец, существует способ дублировать знаки операций, который мы приведём безо всяких пояснений. Включив
%\newcommand*{\hm}[1]{#1\nobreak\discretionary{}{\hbox{\mathsurround=0pt #1}}{}}
%в преамбулу, можно написать $a\hm+b\hm+c\hm+d$, при этом в формуле a\hm+b\hm+c\hm+d при переносе знак + будет продублирован.

% Дублирование символов бинарных операций ("+", "-", "="), набранных в строчных формулах, при переносе на другую строку:
%%begin{latexonly}
%\renewcommand\ne{\mathchar"3236\mathchar"303D\nobreak
%      \discretionary{}{\usefont
%      {OMS}{cmsy}{m}{n}\char"36\usefont
%      {OT1}{cmr}{m}{n}\char"3D}{}}
%\begingroup
%\catcode`\+\active\gdef+{\mathchar8235\nobreak\discretionary{}%
% {\usefont{OT1}{cmr}{m}{n}\char43}{}}
%\catcode`\-\active\gdef-{\mathchar8704\nobreak\discretionary{}%
% {\usefont{OMS}{cmsy}{m}{n}\char0}{}}
%\catcode`\=\active\gdef={\mathchar12349\nobreak\discretionary{}%
% {\usefont{OT1}{cmr}{m}{n}\char61}{}}
%\endgroup
%\def\cdot{\mathchar8705\nobreak\discretionary{}%
% {\usefont{OMS}{cmsy}{m}{n}\char1}{}}
%\def\times{\mathchar8706\nobreak\discretionary{}%
% {\usefont{OMS}{cmsy}{m}{n}\char2}{}}
%\mathcode`\==32768
%\mathcode`\+=32768
%\mathcode`\-=32768
%%end{latexonly}

\makeindex

\begin{document}
\selectlanguage{russian}

%\layout

% рамка границ страницы http://www.ctan.org/tex-archive/macros/latex/contrib/fancybox/fancybox-doc.pdf
% сделать поиск по fancypage, thisfancypage
%\thisfancypage{}{\fbox}
%\thisfancypage{\fbox}{}
%\fancypage{}{\fbox}         % закомментировать
%\fancypage{\fbox}{\fbox}    % закомментировать
%\fancypage{\setlength{\fboxsep}{32pt}\fbox}{}

\title{Защита информации \\ Учебное пособие}
\author{Габидулин Эрнст Мухамедович \\ Кшевецкий Александр Сергеевич \\ Колыбельников Александр Иванович \\ Владимиров Сергей Михайлович}
\date{
 %   \textbf{\textsc{Черновой вариант. Может содержать ошибки.}} \\
%    \today
}
\maketitle
\setcounter{page}{3}

\newpage
%\thispagestyle{empty}
\setcounter{tocdepth}{2}
\tableofcontents
%\thispagestyle{empty}
\newpage

%\lhead[\leftmark]{}
%\rhead[]{\rightmark}

\input{foreword}

\input{Short_history_of_cryptography}

\chapter{Основные понятия и определения}
\selectlanguage{russian}

Изучение курса <<Защита информации>> необходимо начать с определения понятия \emph{<<информация>>}. В теоретической информатике \emph{информация} -- это любые сведения, или цифровые данные, или сообщения, или документы, или файлы, которые могут быть переданы \emph{получателю информации} от \emph{источника информации}. Можно считать, что информация передаётся по какому-либо каналу связи с помощью некоторого носителя, которым может быть, например, распечатка текста, диск или другое устройство хранения информации, система передачи сигналов по оптическим, проводным линиям или радиолиниям связи и~т.\,д.

\emph{Защита информации} -- это\footnote{Строго говоря, определение защиты информации даётся в официальном стандарте ГОСТ Р 50922-2006, <<Защита информации. Основные термины и определения>>~\cite{GOST-50922-2006}, согласно которому \emph{защита информации} -- это деятельность, направленная на предотвращение утечки защищаемой информации, несанкционированных и непреднамеренных воздействий на защищаемую информацию. Однако мы пользуемся определением, основанным на понятии <<безопасность информации>> из того же стандарта: \emph{безопасность информации} -- это состояние защищенности информации, при котором обеспечиваются ее \emph{конфиденциальность}, \emph{доступность} и \emph{целостность}.} обеспечение \emph{целостности}, \emph{конфиденциальности} и \emph{доступности} информации, передаваемой или хранимой в какой-либо форме. Информацию необходимо защищать от нарушения её целостности и конфиденциальности в результате вмешательства \emph{нелегального пользователя}. В российском стандарте ГОСТ Р 50.1.056-2005 приведены следующие определения~\cite{GOST-2005}:
\begin{itemize}
	\item \emph{целостность информации}\index{целостность} -- состояние информации, при котором отсутствует любое ее изменение либо изменение осуществляется только преднамеренно субъектами, имеющими на него право;
	\item \emph{конфиденциальность}\index{конфиденциальность} -- состояние информации, при котором доступ к ней осуществляют только субъекты, имеющие на него право;
	\item \emph{доступность}\index{доступность} -- состояние информации, при котором субъекты, имеющие права доступа, могут реализовать их беспрепятственно.
\end{itemize}

Другой стандарт ГОСТ Р ИСО/МЭК 13335-1-2006~\cite{GOST-13335-1-2006} определяет \emph{информационную безопасность} как все аспекты, связанные с определением, достижением и поддержанием \emph{конфиденциальности}, \emph{целостности}, \emph{доступности}, \emph{неотказуемости}, \emph{подотчетности}, \emph{аутентичности} и \emph{достоверности} информации или средств ее обработки. То есть в дополнение к предыдущему определению на защиту информации в области информационных технологий возлагаются дополнительные задачи:
\begin{itemize}
	\item обеспечение \emph{неотказуемости} -- способность удостоверять имевшее место действие или событие так, чтобы эти события или действия не могли быть позже отвергнуты;
	\item обеспечение \emph{подотчетности} -- способность однозначно прослеживать действия любого логического объекта;
	\item обеспечение \emph{аутентичности} -- способность гарантировать, что субъект или ресурс идентичны заявленным;\footnote{Аутентичность применяется к таким субъектам, как пользователи, к процессам, системам и информации.}
	\item обеспечение \emph{достоверности} -- способность обеспечивать соответствие предусмотренному поведению и результатам;
\end{itemize}

Стандарт ГОСТ Р 50922-2006~\cite{GOST-50922-2006}, хотя и не вводит прямой классификации методов защиты информации, даёт следующие их определения.
\begin{itemize}
	\item \emph{Правовая защита информации}. Защита информации правовыми методами, включающая в себя разработку законодательных и нормативных правовых документов (актов), регулирующих отношения субъектов по защите информации, применение этих документов (актов), а также надзор и контроль за их исполнением.
	\item \emph{Техническая защита информации; ТЗИ}. Защита информации, заключающаяся в обеспечении некриптографическими методами безопасности информации (данных), подлежащей (подлежащих) защите в соответствии с действующим законодательством, с применением технических, программных и программно-технических средств.
	\item \emph{Криптографическая защита информации}. Защита информации с помощью ее криптографического преобразования.
	\item \emph{Физическая защита информации}. Защита информации путем применения организационных мероприятий и совокупности средств, создающих препятствия для проникновения или доступа неуполномоченных физических лиц к объекту защиты.
\end{itemize}

В рамках данного пособия в основном остановимся на криптографических методах защиты информации. Они помогают обеспечить \emph{конфиденциальность} и \emph{аутентичность}. В сочетании с правовыми методами зашиты информации они помогают обеспечить \emph{неотказуемость} действий, а в сочетании с техническими -- \emph{целостность информации} и \emph{достоверность}.

При изучении криптографических методов защиты информации используются дополнительные определения. В целом науку об создании, анализе и использовании криптографических методов называют \emph{криптологией}. Её разделяют на \emph{криптографию}, посвящённую разработке и применению криптографических методов, и \emph{криптоанализ}, который занимается поиском уязвимостей в существующих методах. Данное разделение на криптографию и криптоанализ (и, соответственно, разделение на \emph{криптографов} и \emph{криптоаналитиков}) условно, так как создать хороший криптографический метод невозможно без умения анализировать его потенциальные уязвимости, а поиск уязвимостей в современных криптографических методах нельзя осуществить без знания методов их построения.

Попытка криптоаналитика нарушить свойство криптографической системы по обеспечению защиты информации (например, получить информацию вопреки свойству обеспечения конфиденциальности) называется \emph{криптографической атакой} (\emph{криптоатакой}). Если данная попытка оказалась успешной, и свойство было нарушено или может быть нарушено в ближайшем будущем, то такое событие называется \emph{взломом криптосистемы} или \emph{вскрытием криптосистемы}. Конкретный метод криптографической атаки также называется \emph{криптоанализ} (например, линейный криптоанализ, дифференциальный криптоанализ, и~т.~д.). Криптосистема называется \emph{криптостойкой}, если число стандартных операций для её взлома превышает возможности современных вычислительных средств в течение всего времени ценности информации (до 100 лет).

Для многих криптографических примитивов существует атака полным перебором\index{атака!полным перебором}, либо аналогичная, которая подразумевает, что если выполнить очень большое количество определённых операций (по одной на каждое значение из области определения одного из аргументов криптографического метода), то один из результатов укажет непосредственно на способ взлома системы (например, укажет на ключ для нарушения конфиденциальности, обеспечиваемой алгоритмом шифрования, или на допустимый прообраз для функции хэширования, приводящий к нарушению аутентичности и целостности). В этом случае под \emph{взломом криптосистемы} понимается построение алгоритма криптоатаки с количеством операций меньшим, чем планировалось при создании этой криптосистемы (часто, но не всегда, это равно именно количеству операций при атаке полным перебором\footnote{Например, сложность построения второго прообраза для хэш-функций на основе конструкции Меркла~---~Дамгарда составляет $2^n / \left|M\right|$ операций, тогда как полный перебор -- $2^n$. См. раздел~\ref{section-stribog}}). Взлом криптосистемы – это не обязательно, например, реально осуществленное извлечение информации, так как количество операций может быть вычислительно недостижимым как в настоящее время, так и в течение всего времени защиты. То есть могут существовать системы, которые формально взломаны, но пока ещё являются криптостойкими.

Далее рассмотрим модель передачи информации с отдельными криптографическими методами.

\input{model_of_the_transmission_system_with_crypto}

\section[Классификация]{Классификация криптографических механизмов}

\input{classification_by_symmetry}
\subsection{Шифры замены и перестановки}

Шифры по способу преобразования открытого текста в шифротекст разделяются на шифры замены и шифры перестановки.

\input{substitution_ciphers}

\input{permutation_ciphers}

\input{composite_ciphers}

\subsection{Примеры современных криптографических примитивов}

Приведём примеры названий некоторых современных криптографических примитивов, из которых строят системы защиты информации.
\begin{itemize}
    \item DES\index{шифр!DES}, AES, ГОСТ 28147-89, Blowfish\index{шифр!Blowfish}, RC5\index{шифр!RC5}, RC6\index{шифр!RC6} -- блочные симметричные шифры, скорость обработки -- десятки мегабайт в секунду.
    \item A5/1, A5/2, A5/3\index{шифр!A5}, RC4\index{шифр!RC4} -- потоковые симметричные шифры с высокой скоростью. Семейство A5 применяется в мобильной связи GSM, RC4 -- в компьютерных сетях для SSL-соединения между браузером и веб-сервером.
    \item RSA\index{шифр!RSA} -- криптосистема с открытым ключом для шифрования.
    \item RSA\index{электронная подпись!RSA}, DSA\index{электронная подпись!DSA}, ГОСТ Р 34.10-2001\index{электронная подпись!ГОСТ Р 34.10-2001} -- криптосистемы с открытым ключом для электронной подписи.
    \item MD5\index{хэш-функция!MD5}, SHA-1\index{хэш-функция!SHA-1}, SHA-2\index{хэш-функция!SHA-2}, ГОСТ Р 34.11-94\index{хэш-функция!ГОСТ Р 34.11-94} -- криптографические хэш-функции.
\end{itemize}

\input{Cryptanalysis_methods_and_types_of_attacks}

\input{The_minimum_key_lengths}


\chapter{Классические шифры}

В главе приведены наиболее известные \emph{классические} шифры, которыми можно было пользоваться до появления роторных машин. К ним относятся такие шифры, как шифр Цезаря\index{шифр!Цезаря}, шифр Плейфера\index{шифр!Плейфера}, шифр Хилла\index{шифр!Хилла} и шифр Виженера\index{шифр!Виженера}. Они наглядно демонстрируют различные классы шифров.

\input{monoalphabetic_ciphers}

\input{bigrammnye_substitution_ciphers}

\input{hills_cipher}

% \subsection{Омофонные замены}
%
% Омофонными заменами называют криптопримитивы, в основе которых лежит замена групп символов открытого текста $M$ на группу символов $C$ с использованием ключа $K$. Такой метод шифрования вносит неоднозначность между $M$ и $C$, это позволяет защититься от методов частотного криптоанализа.
%  \subsection{шифрокоды}
%  Шифрокоды -- это класс шифров сочетающих в себе свойства кодов и помехозащищённости со свойствами шифра и обеспечения конфиденциальности.

\input{vigeneres_cipher}

\input{polyalphabetic_cipher_cryptanalysis}

\input{perfect_secure_systems}

\chapter{Блочные шифры}\label{chapter-block-ciphers}\index{шифр!блочный|(}

\input{block_ciphers}

\input{lucifer}

\input{Feistel_cipher}

\input{DES}

\input{GOST_28147-89}

\input{AES}

\section{Шифр «Кузнечик»}\label{section-grig}\index{шифр!«Кузнечик»|(}
\selectlanguage{russian}

В июне 2015 года в России был принят новый стандарт блочного шифрования ГОСТ Р 34.12-2015~\cite{GOST-R:34.12-2015}. Данный стандарт включает в себя два блочных шифра -- старый ГОСТ 28147-89, получивший теперь название <<Магма>>, и новый шифр со 128 битным входным блоком, получившим название <<Кузнечик>>.

В отличие от шифра <<Магма>> новый шифр <<Кузнечик>> основан на SP-сети\index{SP-сеть} (сети замен и перестановок), то есть основан на серии обратимых преобразований, а не на ячейке Фейстеля\index{ячейка Фейстеля}. Как и другие популярные шифры, он является блочным раундовым шифром и имеет выделенную процедуру выработки раундовых ключей. Шифр работает с блоками открытого текста по 128 бит, а размер ключа шифра составляет 256 бит. Отдельный раунд шифра <<Кузнечик>> состоит из операции наложения ключа, нелинейного и линейного преобразований, как изображено на рис.~\ref{fig:kuznechik-step}. Всего в алгоритме 10 раундов, последний из которых состоит только из операции наложения ключа. 

\begin{figure}[htb]
	\centering
	\includegraphics[width=0.8\textwidth]{pic/kuznechik-step}
  \caption{Один раунд шифрования в алгоритме <<Кузнечик>>}
  \label{fig:kuznechik-step}
\end{figure}

Нелинейное преобразование $S$ разбивает блок данных из 128 бит на 16 блоков по 8 бит в каждом, как показано на рис.~\ref{fig:kuznechik-s}.

\begin{figure}[htb]
	\centering
	\includegraphics[width=0.9\textwidth]{pic/kuznechik-s}
  \caption{Нелинейное преобразование $S$ в алгоритме <<Кузнечик>>}
  \label{fig:kuznechik-s}
\end{figure}

Каждый из 8-ми битных блоков $a$ трактуется как целое беззнаковое число $\text{Int}_8 a$ и выступает в качестве индекса в заданном массиве констант $\pi'$. Значение по индексу $\text{Int}_8 a$ в массиве констант $\pi'$ обратно преобразуется в двоичный вид и выступает в качестве одного из 16-ти выходных блоков нелинейного преобразования $S$.

\begin{quote}\noindent {\scriptsize$\pi'$ = (252, 238, 221, 17, 207, 110, 49, 22, 251, 196, 250, 218, 35, 197, 4, 77, 233, 119, 240, 219, 147, 46, 153, 186, 23, 54, 241. 187, 20, 205, 95, 193, 249, 24, 101, 90, 226, 92, 239, 33, 129, 28, 60, 66, 139, 1, 142, 79, 5, 132, 2, 174, 227, 106, 143, 160, 6, 11, 237, 152, 127, 212, 211, 31, 235, 52, 44, 81, 234, 200, 72, 171, 242, 42, 104, 162, 253, 58, 206, 204, 181, 112, 14, 86, 8, 12, 118, 18, 191, 114, 19, 71, 156, 183, 93, 135, 21, 161, 150, 41, 16, 123, 154, 199, 243, 145, 120, 111, 157, 158, 178, 177, 50, 117, 25, 61, 255, 53, 138, 126, 109, 84, 198, 128, 195, 189, 13, 87, 223, 245, 36, 169, 62, 168, 67, 201, 215, 121, 214, 246, 124, 34, 185, 3, 224, 15, 236, 222, 122, 148, 176, 188, 220, 232, 40, 80, 78, 51, 10, 74, 167, 151, 96, 115, 30, 0, 98, 68, 26, 184, 56, 130, 100, 159, 38, 65, 173, 69, 70, 146, 39, 94, 85, 47, 140, 163, 165, 125, 105, 213, 149, 59, 7, 88, 179, 64, 134, 172, 29, 247, 48, 55, 107, 228, 136, 217, 231, 137, 225, 27, 131, 73, 76, 63, 248, 254, 141, 83, 170, 144, 202, 216, 133, 97, 32, 113, 103, 164, 45, 43, 9, 91, 203, 155, 37, 208, 190, 229, 108, 82, 89, 166, 116, 210, 230, 244, 180, 192, 209, 102, 175, 194, 57, 75, 99, 182).}\end{quote}

Линейное преобразование $L$ состоит из 16-ти операций линейного преобразования $R$, то есть $L = R^{16}$. Линейное преобразование $R$, в свою очередь, представляет блок из 128 бит как начальные значения 8-ми битовых ячеек регистра сдвига с линейной обратной связью (РСЛОС) с 16 ячейками, как показано на рис.~\ref{fig:kuznechik-p}. При сдвиге вычисляется сумма значений ячеек, домноженных на 16 констант. Значения ячеек и константы трактуются как элементы поля Галуа $GF(2^8)$ с модулем $p(x) = x^8 + x^7 + x^6 + x + 1$ (см. раздел~\ref{section-fields}), умножение и сложение также проходят в этом поле.

\begin{figure}[htb]
	\centering
	\includegraphics[width=0.9\textwidth]{pic/kuznechik-p}
  \caption{Линейное преобразование $R$ в алгоритме <<Кузнечик>>}
  \label{fig:kuznechik-p}
\end{figure}

Алгоритм развёртывания ключа основан на ячейке Фейстеля, хотя и не использует её ключевую особенность (обратимость). Начало алгоритма изображено на рис.~\ref{fig:kuznechik-keys}.

\begin{figure}[htb]
	\centering
	\includegraphics[width=0.90\textwidth]{pic/kuznechik-keys}
  \caption{Часть алгоритма развёртывания ключа в <<Кузнечике>>}
  \label{fig:kuznechik-keys}
\end{figure}

\begin{itemize}
	\item Целые числа $i$ от 1 до 32 представляются в виде двоичных векторов по 128 бит. К каждому из них применяется линейное преобразование $L=R^{16}$ как было описано ранее. Получаются 32 константы $C_{1}...C_{32}$.
	\item Первые два раундовых ключа $K_1$ и $K_2$ получаются разбиением мастер-ключа $K$ (256 бит) на два блока по 128 бит каждый.
	\item Следующая пара раундовых ключей $K_3$ и $K_4$ получается из первой пары $K_1$ и $K_2$ применением 8 раундов ячейки Фейстеля. В качестве функции Фейстеля, преобразующую один из блоков на каждом раунде, выступает преобразование $\text{LSX}(C_i), i=1,\dots,8$. То есть (читая справа налево, как принято с операторами) побитовое сложение с заданной константой $C_i$, а потом нелинейное и линейное преобразования $S$ и $L$, как они были описаны ранее.
	\item Все остальные пары раундовых ключей вплоть до $K_{9}$ и $K_{10}$ получаются аналогичным образом (использованием предыдущей пары ключей и 8-ми констант $C_i$).
\end{itemize}

Так как и легальный отправитель, и легальный получатель используют функцию развёртывания ключа в прямом направлении, начиная с пары $K_1, K_2$ и до $K_{9}, K_{10}$, то алгоритм никогда не <<идёт назад>> и не использует ключевую особенность ячейки Фейстеля -- её обратимость.

В отличие от стандарта 1989 года, в текст нового стандарта не стали включать режимы сцепления блоков, а вынесли это в отдельный ГОСТ Р 34.13-2015 <<Режимы работы блочных шифров>>~\cite{GOST-R:34.13-2015}.

В работе 2015 года Бирюков, Перрин и Удовенко (\langen{Alex Biryukov, Léo Perrin, Aleksei Udovenko},~\cite{Biryukov:Perrin:Udovenko:2015}) продемонстрировали, что структура s-блока не является случайной, а получена в результате работы детерминированного алгоритма. Это может быть использовано для создания более быстрых реализаций алгоритма шифрования, но теоретически может быть и основой для атак на шифр.

\index{шифр!«Кузнечик»|)}


\section{Режимы работы блочных шифров}\label{section-block-chaining}
\selectlanguage{russian}

Открытый текст $M$, представленный как двоичный файл, перед шифрованием разбивают на части $M_1, M_2, \dots, M_n$, называемые пакетами. Предполагается, что размер в битах каждого пакета существенно превосходит длину блока шифрования, которая равна 64 битам для российского стандарта и 128 -- для американского стандарта AES.

В свою очередь, каждый пакет $M_i$ разбивается на блоки размера, равного размеру блока шифрования:
    \[ M_i = \left[ M_{i,1}, M_{i,2}, \dots, M_{i,n_i} \right]. \]
Число блоков $n_i$ в разных пакетах может быть разным. Кроме того, последний блок пакета $M_{i,n_i}$ может иметь размер, меньший размера блока шифрования. В этом случае для него применяют процедуру дополнения (удлинения) до стандартного размера. Процедура должна быть обратимой: после расшифрования последнего блока пакета лишние байты должны быть обнаружены и удалены. Некоторые способы дополнения:
\begin{itemize}
  \item добавить один байт со значением $128$, а остальные байты принять за нулевые;
  \item определить, сколько байтов надо добавить к последнему блоку, например $b$, и добавить $b$ байтов со значением $b$ в каждом.
\end{itemize}
В дальнейшем предполагается, что такое дополнение сделано для каждого пакета. При шифровании блоков внутри одного пакета первый индекс в нумерации блоков опускается, то есть вместо обозначения $M_{i,j}$ используется $M_j$.

Для шифрования всего открытого текста $M$ и, следовательно, всех пакетов используется один и тот же \emph{сеансовый} ключ шифрования $K$. Процедуру передачи одного пакета будем называть \emph{сеансом}.

Существует несколько режимов работы блочных шифров: режим электронной кодовой книги, режим шифрования сцепленных блоков, режим обратной связи, режим шифрованной обратной связи, режим счётчика. Рассмотрим особенности каждого из этих режимов.


\subsection{Электронная кодовая книга}

В режиме электронной кодовой книги (\langen{Electronic Code Book, ECB}) открытый текст в пакете разделён на блоки
    \[ \left[ M_1, M_2, \dots, M_{n-1}, M_n \right]. \]

В процессе шифрования каждому блоку $M_j$ соответствует свой шифротекст $C_j$, определяемый с помощью ключа $K$:
    \[ C_j = E_K(M_j), ~ j = 1, 2, \dots, n. \]

Если в открытом тексте есть одинаковые блоки, то в шифрованном тексте им также соответствуют одинаковые блоки. Это даёт дополнительную информацию для криптоаналитика, что является недостатком этого режима. Другой недостаток состоит в том, что криптоаналитик может подслушивать, перехватывать, переставлять, воспроизводить ранее записанные блоки, нарушая конфиденциальность\index{конфиденциальность} и целостность\index{целостность} информации. Поэтому при работе в режиме электронной кодовой книги нужно вводить аутентификацию сообщений.

Шифрование в режиме электронной кодовой книги не использует сцепление блоков и синхропосылку\index{синхропосылка} (вектор инициализации)\index{вектор инициализации}. Поэтому для данного режима применима атака на различение сообщений, так как два одинаковых блока или два одинаковых открытых текста шифруются идентично.

На рис.~\ref{fig:ecb-demo} приведён пример шифрования графического файла морской звезды в формате BMP, 24 бит цветности на пиксель (рис.~\ref{fig:starfish}), блочным шифром AES с длиной ключа 128 бит в режиме электронной кодовой книги (рис.~\ref{fig:starfish-aes-128-ecb}). В начале зашифрованного файла был восстановлен стандартный заголовок формата BMP. Как видно, в зашифрованном файле изображение всё равно различимо.
\begin{figure}[!ht]
    \centering
    \subcaptionbox{Исходный рисунок\label{fig:starfish}}{ \includegraphics[width=0.45\textwidth]{pic/starfish}}
    ~~~
    \subcaptionbox{Рисунок, зашифрованный AES-128\label{fig:starfish-aes-128-ecb}}{ \includegraphics[width=0.45\textwidth]{pic/starfish-aes-128-ecb}}
    \caption{Шифрование в режиме электронной кодовой книги\label{fig:ecb-demo}}
\end{figure}
BMP файл в данном случае содержит в самом начале стандартный заголовок (ширина, высота, количество цветов), и далее идёт массив 24-битовых значений цвета пикселей, взятых построчно сверху вниз. В массиве много последовательностей нулевых байтов, так как пиксели белого фона кодируются 3 нулевыми байтами. В AES размер блока равен 16 байтам, и, значит, каждые $\frac{16}{3}$ подряд идущих пикселей белого фона шифруются одинаково, позволяя различить изображение в зашифрованном файле.

%На рис.~\ref{fig:ecb-demo} приведён пример шифрования графического файла логотипа Википедии в формате BMP, 24 бит цветности на пиксель (рис.~\ref{fig:wikilogo}), блочным шифром AES с длиной ключа 128 бит в режиме электронной кодовой книги (рис.~\ref{fig:wikilogo-aes-128-ecb}). В начале зашифрованного файла был восстановлен стандартный заголовок BMP формата. Как видно, на зашифрованном рисунке возможно даже прочитать надпись.
%\begin{figure}[!ht]
%    \centering
%    \subfloat[Исходный рисунок]{\label{fig:wikilogo}\includegraphics[width=0.45\textwidth]{pic/wikilogo}}
%    ~~~
%    \subfloat[Рисунок, зашифрованный AES-128]{\label{fig:wikilogo-aes-128-ecb}\includegraphics[width=0.45\textwidth]{pic/wikilogo-aes-128-ecb}}
%    \caption{Шифрование в режиме электронной кодовой книги.}
%    \label{fig:ecb-demo}
%\end{figure}

%Возможно воссоздание структуры информации -- например, пингвин на рис.~\ref{fig:tux-ecbmode}. Картинка с пингвином записана в формате BMP и зашифрована DES в режиме электронной кодовой книги.
%\begin{figure}[!ht]
%    \centering
%    \includegraphics[width=0.3\textwidth]{pic/tux-ecb}
%    \caption{Картинка с пингвином, зашифрованная в режиме электронной кодовой книги.}
%    \label{fig:tux-ecbmode}
%\end{figure}


\subsection{Сцепление блоков шифротекста}

В режиме сцепления блоков шифротекста (\langen{Cipher Block Chaining, CBC}) перед шифрованием текущего блока открытого текста предварительно производится его суммирование по модулю $2$ с предыдущим блоком зашифрованного текста, что и осуществляет <<сцепление>> блоков. Процедура шифрования имеет вид:
\[ \begin{array}{l}
    C_1 = E_K(M_1 \oplus C_0), \\
    C_j = E_K(M_j \oplus C_{j-1}), ~ j = 1, 2, \dots, n,
\end{array} \]
где $C_0 = \textrm{IV}$ -- вектор, называемый вектором инициализации (обозначение $\textrm{IV}$ от Initialization Vector). Другое название -- синхропосылка.

Благодаря сцеплению, \emph{одинаковым} блокам открытого текста соответствуют \emph{различные} шифрованные блоки. Это затрудняет криптоаналитику статистический анализ потока шифрованных блоков.

На приёмной стороне расшифрование осуществляется по правилу
\[ \begin{array}{l}
    D_K(C_j) = M_j \oplus C_{j-1}, ~ j=1, 2, \dots, n,\\
    M_{j} = D_K(C_j) \oplus C_{j-1}.
\end{array} \]

Блок $C_0 = \textrm{IV}$ должен быть известен легальному получателю шифрованных сообщений. Обычно криптограф выбирает его случайно и вставляет на первое место в поток шифрованных блоков. Сначала передают блок $C_0$, а затем шифрованные блоки $C_1, C_2, \ldots, C_n$.

В разных пакетах блоки $C_0$ должны выбираться независимо. Если их выбрать одинаковыми, то возникают проблемы, аналогичные проблемам в режиме ECB. Например, часто первые нешифрованные блоки $M_1$ в разных пакетах бывают одинаковыми. Тогда одинаковыми будут и первые шифрованные блоки.

Однако случайный выбор векторов инициализации также имеет свои недостатки. Для выбора такого вектора необходим хороший генератор случайных чисел. Кроме того, каждый пакет удлиняется на один блок.

Нужны такие процедуры выбора $C_0$ для каждого сеанса передачи пакета, которые известны криптографу и легальному пользователю. Одним из решений является использование так называемых \emph{одноразовых меток}. Каждому сеансу присваивается уникальное число. Его уникальность состоит в том, что оно используется только один раз и никогда не должно повторяться в других пакетах. В англоязычной научной литературе оно обозначается как \emph{Nonce}, то есть сокращение от <<Number used once>>\index{одноразовая метка}.

Обычно одноразовая метка состоит из номера сеанса и дополнительных данных, обеспечивающих уникальность. Например, при двустороннем обмене шифрованными сообщениями одноразовая метка может состоять из номера сеанса и индикатора направления передачи. Размер одноразовой метки должен быть равен размеру шифруемого блока. После определения одноразовой метки $\textrm{Nonce}$ вектор инициализации вычисляется в виде
    \[ C_0 = \textrm{IV} = E_K(\textrm{Nonce}). \]

Этот вектор используется в данном сеансе для шифрования открытого текста в режиме CBC. Заметим, что блок $C_0$ передавать в сеансе не обязательно, если приёмная сторона знает заранее дополнительные данные для одноразовой метки. Вместо этого достаточно вначале передать только номер сеанса в открытом виде. Принимающая сторона добавляет к нему дополнительные данные и вычисляет блок $C_0$, необходимый для расшифрования в режиме CBC. Это позволяет сократить издержки, связанные с удлинением пакета. Например, для шифра AES длина блока $C_0$ равна $16$ байтов. Если число сеансов ограничить величиной $2^{32}$ (вполне приемлемой для большинства приложений), то для передачи номера пакета понадобится только $4$ байта.


\subsection{Обратная связь по выходу}

В предыдущих режимах входными блоками для устройств шифрования были непосредственно блоки открытого текста.
В режиме обратной связи по выходу (OFB от Output FeedBack) блоки открытого текста непосредственно на вход устройства шифрования не поступают. Вместо этого устройство шифрования генерирует псевдослучайный поток байтов, который суммируется по модулю $2$ с открытым текстом для получения шифрованного текста. Шифрование осуществляют по правилу:
\[ \begin{array}{l}
    K_0 = \textrm{IV}, \\
    K_j = E_K(K_{j-1}), ~ j = 1, 2, \dots, n, \\
    C_j = K_j \oplus M_j.
\end{array} \]

Здесь текущий ключ $K_j$ есть результат шифрования предыдущего ключа $K_{j-1}$. Начальное значение $K_0$ известно криптографу и легальному пользователю. На приёмной стороне расшифрование выполняют по правилу:
\[ \begin{array}{l}
    K_0 = \textrm{IV}, \\
    K_j = E_K(K_{j-1}), ~ j = 1, 2, \dots, n, \\
    M_j = K_j \oplus C_j.
\end{array} \]

Как и в режиме CBC, вектор инициализации $\textrm{IV}$ может быть выбран случайно и передан вместе с шифрованным текстом, либо вычислен на основе одноразовых меток. Здесь особенно важна уникальность вектора инициализации.

Достоинство этого режима состоит в полном совпадении операций шифрования и расшифрования. Кроме того, в этом режиме не надо проводить операцию дополнения открытого текста.


\subsection{Обратная связь по шифрованному тексту}

В режиме обратной связи по шифрованному тексту (CFB от Cipher FeedBack) ключ $K_j$ получается с помощью процедуры шифрования предыдущего шифрованного блока $C_{j-1}$. Может быть использован не весь блок $C_{j-1}$, а только его часть. Как и в предыдущем случае, начальное значение ключа $K_0$ известно криптографу и легальному пользователю:
\[ \begin{array}{l}
    K_0 = \textrm{IV}, \\
    K_j = E_K(C_{j-1}), ~ j = 1, 2, \dots, n,\\
    C_j = K_j \oplus M_j.
\end{array} \]

У этого режима нет особых преимуществ по сравнению с другими режимами.


\subsection{Счётчик}

В режиме счётчика (CTR от Counter) правило шифрования имеет вид, похожий на режим обратной связи по выходу (OFB), но позволяющий вести независимое (параллельное) шифрование и расшифрование блоков:
\[ \begin{array}{l}
    K_j = E_K(\textrm{Nonce} ~\|~ j - 1), ~ j = 1, 2, \dots, n, \\
    C_j = M_j \oplus K_j,
\end{array} \]
где $\textrm{Nonce} ~\|~ j - 1$ -- конкатенация битовой строки одноразовой метки $\textrm{Nonce}$ и номера блока, уменьшенного на единицу.
%Для стандарта AES значение $\textrm{Nonce}$ занимает 16 бит, номер блока -- 48 бит. С одним ключом выполняется шифрование $2^{48}$ блоков.

Правило расшифрования идентичное:
\[ \begin{array}{l}
    M_j = C_j \oplus K_j. \\
\end{array} \]


\section{Некоторые свойства блочных шифров}

\input{feistel_network_reversibility}

\input{Feistel_cipher_without_s_blocks}

\input{Avalanche_effect}

\input{double_and_triple_ciphering}

\index{шифр!блочный|)}

\input{generators}

\input{stream-ciphers}

\input{hash-functions}

\input{public-key}

\chapter{Распространение ключей}\index{протокол!распространения ключей}\label{chapter-key-distribution-protocols}
\selectlanguage{russian}

Задачей распространения ключей между двумя пользователями является создание секретных псевдослучайных сеансовых ключей шифрования и аутентификация сообщений. Пользователи предварительно создают и обмениваются ключами аутентификации один раз. В дальнейшем для создания защищённой связи пользователи производят взаимную аутентификацию и вырабатывают сеансовые ключи\index{ключ!сеансовый}.

\section[Трёхэтапный протокол Шамира]{Трёхэтапный протокол Шамира на коммутативных шифрах}
\selectlanguage{russian}

Предположим, что две стороны $A$ и $B$ соединены незащищённым каналом связи. Каждая из этих сторон имеет свой секретный ключ: $A$ имеет ключ $K_A$, $B$ имеет ключ $K_B$. Сторона $A$ должна создать общий секретный ключ $K$ и передать стороне $B$.

Для решения этой задачи используют трёхэтапный протокол Шамира с тремя <<замками>>. \emph{Протокол Шамира}\index{протокол!Шамира} построен на \emph{коммутативных} функциях шифрования, для которых выполняется условие:
    \[ E_{K_{B}} (E_{K_{A}}(K))=E_{K_{A}} (E_{K_{B}}(K)). \]

Протокол предполагает следующие процедуры:
\begin{enumerate}
    \item $A$ создаёт секретный ключ $K$, шифрует его своей системой шифрования с помощью своего ключа $K_A$ и посылает сообщение стороне $B$:
        \[ A \rightarrow B: ~ E_{K_A}(K). \]
    \item $B$ получает это сообщение, шифрует его с помощью своего ключа $K_B$ и посылает сообщение стороне $A$:
        \[ A \leftarrow B: ~ E_{K_B}( E_{K_A}( K)). \]
    \item Сторона $A$, получив сообщение $E_{K_B}(E_{K_A}(K))$, использует свой секретный ключ $K_A$ для расшифрования:
            \[ D_{K_A}(E_{K_B} (E_{K_A}(K))) = E_{K_B}(K). \]
        Сторона $A$ передаёт стороне $B$ сообщение:
        \[ A \rightarrow B: ~ E_{K_B}(K). \]
    \item Сторона $B$, получив сообщение $E_{K_B}(K)$, использует свой секретный ключ $K_B$ для расшифрования:
            \[ D_{K_B}(E_{K_B}(K)) = K. \]
        В результате стороны получают общий секретный ключ $K$.
\end{enumerate}

Приведём пример неудачного шифрования с использованием коммутативных функций.

\begin{enumerate}
    \item $A$ имеет функцию шифрования совершенной секретности $E_{K_A}(K) = K \oplus K_A$, где $K_A$ -- двоичная последовательность с равномерным распределением символов. $A$ посылает это сообщение стороне $B$:
            \[ A \rightarrow B: ~ E_{K_A}(K) = K \oplus K_A. \]
    \item $B$ использует такую же функцию шифрования совершенной секретности с ключом $K_B$ (двоичная последовательность с равномерным распределением символов). $B$ шифрует полученное сообщение и отправляет $A$:
            \[ A \leftarrow B: ~ E_{K_A}(K) \oplus K_B = K \oplus K_A \oplus K_B. \]
    \item Сторона $A$, получив сообщение $K \oplus K_A \oplus K_B$, выполняет расшифрование:
            \[ K \oplus K_A \oplus K_B \oplus K_A = K \oplus K_B. \]
        Сторона $A$ передаёт стороне $B$ сообщение:
            \[ A \rightarrow B: ~ K \oplus K_B. \]
    \item Сторона $B$, получив сообщение $K \oplus K_B$, выполняет расшифрование:
            \[ K \oplus K_B \oplus K_B = K. \]
        Обе стороны получают общий секретный ключ $K$.
\end{enumerate}

Предложенный выбор коммутативной функции шифрования совершенной секретности был назван неудачным, так как существуют ситуации, при которых криптоаналитик может определить ключ $K$. Предположим, что криптоаналитик перехватил все три сообщения:
    \[ K \oplus K_A, ~~ K \oplus K_A \oplus K_B, ~~ K \oplus K_B. \]
Сложение по модулю 2 всех трёх сообщений даёт ключ $K$. Поэтому такая система шифрования не применяется.

Теперь приведём протокол надёжной передачи секретного ключа, основанный на экспоненциальной (коммутативной) функции шифрования. Стойкость этого протокола связана с трудностью задачи вычисления дискретного логарифма: известны значения $y, g, p$, найти $x$ в уравнении $y = g^x \mod p$.

\textbf{Протокол Шамира распространения ключей}

Выбирают большое простое\index{число!простое} число $p\sim 2^{1024}$ и используют его как открытый ключ.

\begin{enumerate}
    \item Сторона $A$ задаёт общий секретный ключ $K <p$ и выбирает целое число $a$, взаимно простое с $p-1$. $A$ вычисляет и посылает сообщение стороне $B$:
            \[ A \rightarrow B: ~ K^a \mod p. \]
        Существует число $c$ такое, что $a c =1 \mod (p-1)$, то есть $a c = 1 + l (p-1)$, где $l$ -- целое число. Число $c$ будет использовано стороной $A$ на следующем этапе.
    \item Сторона $B$ выбирает целое число $b$, взаимно простое с $p-1$. Используя полученное сообщение, вычисляет и посылает сообщение стороне $A$:
            \[ A \leftarrow B: ~ (K^a)^b \mod p. \]
    \item Сторона $A$, получив сообщение, вычисляет
        \[ \left( K^{ab} \right)^c = K^{(1 + l (p-1)) b} = K^b \cdot K^{l (p-1) b} = K^b \mod p. \]
        Здесь применена малая теорема Ферма\index{теорема!Ферма малая}: $K^{p-1} = 1 \mod p$, поэтому $\left( K^{p-1} \right)^{lb} = 1 \mod p$.
        $A$ посылает $B$ сообщение:
            \[ A \rightarrow B: ~ K^b \mod p. \]
    \item Сторона $B$, получив сообщение $K^{b}\mod p$, вычисляет
        \[ (K^b \mod p)^d = K^{bd} \mod p = K, \]
        где $d$ найдено из $b d =1 \mod (p-1)$.
\end{enumerate}

Теперь проверим криптостойкость этого протокола. Предположим, что криптоаналитик перехватил три сообщения:
\[ \begin{array}{l}
    y_1 = K^a \mod p, \\
    y_2 = K^{ab} \mod p, \\
    y_3 = K^b \mod p. \\
\end{array} \]
Чтобы найти ключ $K$, надо решить систему из этих трёх уравнений, что имеет очень большую вычислительную сложность, неприемлемую с практической точки зрения, если все три числа $a, b, ab$ достаточно велики. Предположим, что $a$ (или $b$) мало. Тогда, вычисляя последовательные степени $y_3$ (или $y_1$), можно найти $a$ (или $b$), сравнивая результат с $y_2$. Зная $a$, легко найти $a^{-1}\mod(p-1)$ и $K=(y_1)^{a^{-1}}\mod p$.

Недостатком этого протокола является отсутствие аутентификации сторон. Следовательно, нужно дополнительно использовать цифровую подпись при передаче сообщения.


\section{Симметричные протоколы}

\subsection{Аутентификация и атаки воспроизведения}

Рассмотрим такую ситуацию: обе стороны $A$ и $B$ имеют общий долговременный ключ $K_{AB}$ и симметричную систему шифрования. Нужно выработать сеансовый секретный ключ $K$. Сторона $A$ создаёт ключ $K$ и желает его передать стороне $B$.

\begin{enumerate}
    \item Для этого сторона $A$ с помощью общего ключа $K_{AB}$ создаёт и передаёт $B$ шифрованное сообщение:
            \[ A \rightarrow B: ~ E_{K_{AB}}(K, B, A). \]
        В этом сообщении имеются так называемые поля -- $(B,A)$ -- информация для дополнительного подтверждения.
    \item Сторона $B$, используя общий ключ $K_{AB}$, расшифровывает полученное сообщение:
            \[ D_{K_{AB}}( E_{K_{AB}}( K, B, A)) = (K, B, A). \]
        В результате сторона $B$ получает сеансовый ключ $K$ и дополнительные данные $(B,A)$.
\end{enumerate}

Недостаток этого протокола состоит в том, что криптоаналитик может перехватывать сообщения и через некоторое время переслать их стороне $A$.

Рассмотрим другие варианты решения задачи о передаче сеансового ключа.
Задача остаётся прежней: обе стороны $A$ и $B$ имеют общий долговременный секретный ключ $K_{AB}$, сторона $A$ должна выработать сеансовый секретный ключ $K$ и доставить его стороне $B$.

Протокол включает \emph{метки времени} -- информацию о моменте $t_A$ отправки сообщения и моменте получения сообщения $t_B$.

\begin{enumerate}
    \item Сторона $A$ вырабатывает $K$, с помощью долговременного ключа $K_{AB}$ создаёт шифрованное сообщение с меткой времени $t_A$ и передаёт его стороне $B$:
            \[ A \rightarrow B: ~ E_{K_{AB}}(K, t_A). \]
    \item Сторона $B$ получает сообщение и расшифровывает его с помощью общего ключа:
            \[ D_{K_{AB}}( E_{K_{AB}}( K, t_A)) = (K, t_A). \]
        В результате $B$ получает $(K, t_A)$, то есть секретный ключ и метку времени. $B$ измеряет время прихода $t_B$ и интервал запаздывания. Если $|t_B - t_A| \leq \delta$, то $B$ аутентифицирует $A$.
\end{enumerate}
Метка времени является одноразовой меткой и защищает от атак воспроизведения ранее записанных сообщений.

Рассмотрим другой способ передачи ключа с дополнительной информацией в виде \emph{одноразовых случайных меток} (nonce -- number used once) вместо меток времени. Протокол передачи состоит в следующем.

\begin{enumerate}
    \item Сторона $A$ вырабатывает случайное число $r_A$, шифрует сообщение, в котором $(r_A, A)$ -- реквизиты $A$, и передаёт его стороне $B$:
            \[ A \rightarrow B: ~ E_{K_{AB}}(r_A, A). \]
    \item Сторона $B$ вырабатывает сеансовый ключ $K$, создаёт шифрованное сообщение и посылает его $A$:
            \[ A \leftarrow B: ~ E_{K_{AB}}(K, r_A, A). \]
    \item Сторона $A$ расшифровывает полученное сообщение:
            \[ D_{K_{AB}}( E_{K_{AB}}( K, r_A, A)) = (K, r_A, A). \]
        В результате $A$ получает сеансовый ключ и подтверждение своих реквизитов, что является дополнительной аутентификацией.
\end{enumerate}

Предположим, что сторона $B$ тоже желает убедиться, что имеет дело со стороной $A$. Тогда этот протокол следует дополнить передачей реквизитов $B$. По-прежнему считаем, что у $A$ и $B$ общая система шифрования с долговременным секретным ключом $K_{AB}$.

\begin{enumerate}
    \item Сторона $A$ вырабатывает случайное число $r_A$, шифрует и передаёт стороне $B$ сообщение, в котором $(r_A, A)$ -- реквизиты $A$:
            \[ A \rightarrow B: ~ E_{K_{AB}}(r_A, A). \]
    \item Сторона $B$ вырабатывает случайное число $r_B$ и отправляет стороне $A$ зашифрованное сообщение:
            \[ A \leftarrow B: ~ E_{K_{AB}}(K_B, r_B, r_A, A), \]
        где $K_B$ -- ключ $B$.
     \item Сторона $A$ осуществляет расшифрование:
            \[ D_{K_{AB}}(E_{K_{AB}}(K_B, r_B, r_A, A)) = (K_B, r_B, r_A, A), \]
        и получает ключ $K_B$ и реквизиты $r_B, r_A, A$. Для аутентификации себя сторона $A$ создаёт свой ключ $K_A$ и отправляет стороне $B$ шифрованное сообщение:
            \[ A \rightarrow B: ~ E_{K_{AB}}(K_A, r_B, r_A, B). \]
     \item Сторона $B$ осуществляет расшифрование:
            \[ D_{K_{AB}}(E_{K_{AB}}(K_A, r_B, r_A, B)) = (K_A, r_B, r_A, B), \]
        которое определяет ключ $K_A$ и аутентифицирует $A$.
\end{enumerate}

Таким образом, обе стороны имеют в своём распоряжении ключи $K_A, K_B$ в качестве сеансовых секретных ключей.

\subsection{Протокол с ключевым кодом аутентификации}

При использовании хэш-функции $K = h(K_{A} ~\|~ K_{B})$ происходит усиление секретности. Здесь $(K_{A} ~\|~ K_{B})$ -- конкатенация $K_{A} $ и $K_{B}$.

% Достоинства: предположим, $K_{A} ,K_{B} $ -- не обладают «хорошими» свойствами случайности (биты распределены неравномерно или зависимы друг от друга), то есть, $P_{K_{A} ,K_{B} } (0)=\frac{1}{2} -\varepsilon $, где $\varepsilon $ - мало, но не 0. Тогда вероятность того, что этот бит в \emph{K }будет равным нулю, $P_{K} (0)=\frac{1}{2} -\varepsilon ',\varepsilon '<\varepsilon $- усиление секретности.

Вычисление хэш-значения, как правило, выполняется быстрее, чем расшифрование. Поэтому были разработаны протоколы, в которых вместо функции шифрования используется имитовставка\index{имитовставка} на основе хэш-функции ($\MAC_K$). Рассмотрим протокол такого рода.
\begin{enumerate}
    \item Сторона $A$ вырабатывает сеансовый ключ $K$, создаёт сообщение, используя одноразовую метку $t_{A}$, и пересылает его стороне $B$:
            \[ A \rightarrow B: ~ t_A, ~ B, ~ K \oplus \MAC_{K_{AB}}( t_A, B), ~ \MAC_{K_{AB}}(K, t_A, B). \]
    \item Сторона $B$ вычисляет
            \[ \MAC_{K_{AB}}(t_A, B) \oplus K \oplus \MAC_{K_{AB}}(t_A, B) = K \]
        и получает сеансовый ключ $K$.
\end{enumerate}

Заметим, что криптоаналитик может добавить в поле случайную последовательность, тогда вместо $K$ получаем <<$K$ плюс помеха>>. Вмешательство криптоаналитика будет выявлено благодаря наличию четвёртого поля в сообщении. Используя полученное значение $K$, сторона $B$ вычисляет $\MAC_{K_{AB}}(K, t_A, B)$ и сравнивают с четвёртым полем. Если совпадает, то вмешательства криптоаналитика не было.

\subsection{Протокол Нидхема~---~Шрёдера с доверенным центром}\index{протокол!Нидхема~---~Шрёдера}
\selectlanguage{russian}

Рассмотрим ситуацию, когда в сети имеется некоторый надёжный (доверенный) сервер (центр) $T$, которому доверяют все пользователи сети. Сервер для работы с абонентами сети использует некоторую систему шифрования $E_S(*)$, где ключ $S=K_{AT}$ известен только $A$ и $T$, и неизвестен остальным участникам сети, $S = K_{BT}$ известен только $B$ и $T$. Предполагаем, что сервер имеет хороший генератор случайных чисел. Сеансовый ключ сервер вырабатывает по запросу. Стороны $A$ и $B$ могут выбирать разные одноразовые метки.

Приведём в качестве примера упрощённую версию известного \emph{протокола Нидхема~---~Шрёдера} (Needham~---~Schroeder) с симметричным шифром.
\begin{enumerate}
    \item Сторона $A$ передаёт серверу $T$ реквизиты сторон $A$ и $B$ и некую одноразовую метку $N_A$, которая может быть, например, меткой времени или случайным (одноразовым) числом, что оговаривается заранее:
            \[ A \rightarrow T: ~ A, B, N_A. \]
    \item Сервер $T$ вырабатывает секретный сеансовый ключ $K$ для $A$ и $B$ и отправляет стороне $A$ шифрованное сообщение:
            \[ A \leftarrow T: ~ E_{K_{AT}}(N_A, B, K, E_{K_{BT}}(K, A)). \]
    \item Сторона $A$ расшифровывает сообщение:
            \[ D_{K_{AT}}( E_{K_{AT}}(N_A, B, K, E_{K_{BT}}(K, A))) = (N_A, B, K, E_{K_{BT}}(K, A)) \]
        и, чтобы доставить ключ, передаёт стороне $B$ сообщение:
            \[ A \rightarrow B: ~ E_{K_{BT}}(K, A). \]
    \item Сторона $B$ расшифровывает полученное сообщение:
            \[ D_{K_{BT}}( E_{K_{BT}}( K,A)) = (K,A) \]
        и получает ключ и реквизиты $A$, которые требуются для того, чтобы сторона $B$ знала, кому отвечать. Кроме того, сторона $B$ дополнительно желает идентифицировать сторону $A$. Для этого $B$ пересылает $A$ зашифрованную одноразовую метку:
            \[ A \leftarrow B: ~ E_{K}(N_B). \]
    \item Сторона $A$ расшифровывает:
            \[ D_K( E_K( N_B)) = N_B \]
        и возвращает $B$ изменённую одноразовую метку:
            \[ A \rightarrow B: ~ E_K(N_B + 1). \]
    \item Сторона $B$ расшифровывает:
            \[ D_K( E_K( N_B + 1)) = N_B + 1, \]
        проверяет $N_B$ и убеждается, что имеет дело со стороной $A$.
    \item Если требуется двусторонняя аутентификация, то аналогично поступают со стороной $A$: на некотором этапе вносится одноразовая метка $N_A$.
\end{enumerate}


\section{Асимметричные протоколы}

Асимметричные протоколы, или же протоколы, основанные на криптосистемах с открытыми ключами, позволяют ослабить требования к предварительному этапу протоколов. Вместо общего секретного ключа, который должны иметь две стороны (либо обе стороны и доверенный центр), в рассматриваемых ниже протоколах стороны должны предварительно обменяться открытыми ключами (между собой либо между собой и доверенным центром). Такой предварительный обмен может проходить по открытому каналу связи, в предположении, что криптоаналитик не может повлиять на содержимое канала связи на данном этапе.

\subsection{Простой протокол}

Рассмотрим протокол распространения ключей с помощью асимметричных шифров. Введём обозначения: $K_B$ -- открытый ключ стороны $B$, а $K_A$ -- открытый ключ стороны $A$. Протокол включает три сеанса обмена информацией.
\begin{enumerate}
    \item В первом сеансе сторона $A$ посылает стороне $B$ сообщение:
            \[ A \rightarrow B: ~ E_{K_B}(K_1, A), \]
        где $K_1$ -- ключ, выработанный стороной $A$.
    \item Сторона $B$ получает $(K_1, A)$ и передаёт стороне $A$ наряду с другой информацией свой ключ $K_2$ в сообщении, зашифрованном с помощью открытого ключа $K_A$:
            \[ A \leftarrow B: ~ E_{K_A}(K_2, K_1, B). \]
    \item Сторона $A$ получает и расшифровывает сообщение $(K_2, K_1, B)$. Во время третьего сеанса сторона $A$, чтобы подтвердить, что она знает ключ $K_2$, посылает стороне $B$ сообщение:
            \[ A \rightarrow B: ~ E_{K_B}(K_2). \]
\end{enumerate}
Общий ключ формируется из двух ключей: $K_1$ и $K_2$.

\subsection{Протоколы с цифровыми подписями}

Существуют протоколы обмена, в которых перед началом обмена ключами генерируются подписи сторон $A$ и $B$, соответственно $S_A(m)$ и $S_B(m)$. В этих протоколах можно использовать различные одноразовые метки. Рассмотрим пример.
\begin{enumerate}
    \item Сторона $A$ выбирает ключ $K$ и вырабатывает сообщение:
            \[ \left( K, ~ t_A, ~ S_A(K, t_A, B) \right), \]
        где $t_A$ -- метка времени. Зашифрованное сообщение передаёт стороне $B$:
        \[ A \rightarrow B: ~ E_{K_B}(K, ~ t_A, ~ S_A(K, t_A, B)). \]
    \item Сторона $B$ получает это сообщение, расшифровывает $\left( K, ~ t_A, ~ S_A(K, t_A, B) \right)$ и вырабатывает свою метку времени $t_B$. Проверка считается успешной, если $|t_B - t_A | < \delta $. Сторона $B$ знает свои реквизиты и может осуществлять проверку подписи.
\end{enumerate}

Имеется второй вариант протокола, в котором шифрование и подпись выполняются раздельно.
\begin{enumerate}
    \item Сторона $A$ вырабатывает ключ $K$, использует одноразовую метку (или метку времени) $t_{A}$ и передаёт стороне $B$ два различных зашифрованных сообщения:
            \[ \begin{array}{ll}
                A \rightarrow B: & ~ E_{K_B}(K, t_A), \\
                A \rightarrow B: & ~ S_A(K, t_A, B). \\
            \end{array} \]
    \item Сторона $B$ получает это сообщение, расшифровывает $K, t_A$ и, добавив свои реквизиты $B$, может проверить подпись $S_A(K, t_A, B)$.
\end{enumerate}

В третьем варианте протокола сначала производится шифрование, потом подпись.
\begin{enumerate}
    \item Сторона $A$ вырабатывает ключ $K$, использует одноразовую случайную метку или метку времени $t_A$ и передаёт стороне $B$ сообщение:
        \[ A \rightarrow B: ~ t_A, ~ E_{K_B}(K, A), ~ S_A(t_A, ~ K, ~ E_{K_B}(K, A)). \]
    \item Сторона $B$ получает это сообщение, расшифровывает $\left( t_A, ~ K, ~ A, ~ E_{K_B}(K, A) \right)$ и проверяет подпись $S_A(t_A, ~ K, ~ E_{K_B}(K, A))$.
\end{enumerate}

\input{diffie-hellman}

%\section{Протоколы с аутентификацией}

\subsection{Односторонняя аутентификация}

\input{el-gamal_protocol}

\input{mti}

\input{sts}

\input{girault_scheme}

\input{bloms_scheme}

В этом разделе были рассмотрены протоколы, в которых ключи вырабатываются в процессе обмена информацией.
%Существует и другой подход, который будет рассмотрен в следующих разделах.


\input{secret-sharing}

\chapter{Примеры систем защиты}

\input{kerberos}

\input{pgp}

\input{tls}

\input{ipsec}

\section[Защита персональных данных в мобильной связи]{Защита персональных данных в \protect\\ мобильной связи}

\input{gsm2}

\input{gsm3}

%\section{Беспроводная сеть Wi-Fi}
%\subsection{WPA-PSK2, 802.11n, Radix?}
%\subsection{Wimax 802.16(?)}

\chapter{Аутентификация пользователя}


\section{Многофакторная аутентификация}

Для защищённых приложений применяется \emph{многофакторная} аутентификация одновременно по факторам различной природы:
\begin{enumerate}
    \item Свойство, которым обладает субъект. Например: биометрия, природные уникальные отличия (лицо, радужная оболочка глаз, папиллярные узоры, последовательность ДНК).
    \item Знание -- информация, которую знает субъект. Например: пароль, PIN-код.
    \item Владение -- вещь, которой обладает субъект. Например: электронная или магнитная карта, флеш-память.
%    \item Факторы присвоения. Например, номер машины, RFID-метка.
\end{enumerate}

В обычных массовых приложениях из-за удобства использования применяется аутентификация только по \emph{паролю}\index{пароль}, который является общим секретом пользователя и информационной системы. Биометрическая аутентификация по отпечаткам пальцев применяется существенно реже. Как правило, аутентификация по отпечаткам пальцев является дополнительным, а не вторым обязательным фактором (тоже из-за удобства её использования).

%Так же явно или неявно используется аутентификация по факторам:
%\begin{enumerate}
%    \item Социальная сеть. Доверие к индивидууму в личном или интернет общении, на основании общих связей.
%    \item Географическое положение. Например, для проверки оплаты товаров по кредитной карте.
%    \item Время. Доступ к сервисам или местам только в определённое время.
%    \item И др.
%\end{enumerate}


\section[Энтропия и криптостойкость паролей]{Энтропия и криптостойкость \protect\\ паролей}

Стандартный набор символов паролей, которые можно набрать на клавиатуре, используя английские буквы и небуквенные символы, состоит из $D=94$ символов. При длине пароля $L$ символов и предположении равновероятного использования символов энтропия паролей равна
    \[ H = L \log_2 D. \]

Клод Шеннон, исследуя энтропию символов английского текста, изучал вероятность успешного предсказания людьми следующего символа по первым нескольким символам слов или текста. В результате Шеннон получил оценку энтропии первого символа $s_1$ текста порядка $H(s_1) \approx 4{,}6$--$4{,}7$ бит/символ и оценки энтропий последующих символов, постепенно уменьшающиеся до $H(s_9) \approx 1{,}5$ бит/символ для 9-го символа. Энтропия для длинных текстов литературных произведений получила оценку $H(s_\infty) \approx 0{,}4$ бит/символ.

Статистические исследования баз паролей показывают, что наиболее часто используются буквы <<a>>, <<e>>, <<o>>, <<r>> и цифра <<1>>.

NIST использует следующие рекомендации для оценки энтропии паролей\index{энтропия!пароля}, создаваемых людьми.
\begin{enumerate}
    \item Энтропия первого символа $H(s_1) = 4$ бит/символ.
    \item Энтропия со 2-го по 8-й символы $H(s_{i}) = 2$ бит/символ, $2 \leq i \leq 8$.
    \item Энтропия с 9-го по 20-й символы $H(s_{i}) = 1{,}5$ бит/символ, $9 \leq i \leq 20$.
    \item Энтропия с 21-го символа $H(s_{i}) = 1$ бит/символ, $i \geq 21$ 
    \item Проверка композиции на использование символов разных регистров и небуквенных символов добавляет до 6 бит энтропии пароля.
    \item Словарная проверка на слова и часто используемые пароли добавляет до 6 бит энтропии для коротких паролей. Для 20-символьных и более длинных паролей прибавка к энтропии 0 бит.
\end{enumerate}

Для оценки энтропии пароля нужно сложить энтропии символов $H(s_i)$ и сделать дополнительные надбавки, если пароль удовлетворяет тестам на композицию и отсутствие в словаре.

\begin{table}[!ht]
    \caption{Оценка NIST предполагаемой энтропии паролей\label{tab:password-entropy}}
    \resizebox{\textwidth}{!}{ \begin{tabular}{|c||c|c|c||c|}
        \hline
        \multirow{2}{*}{\parbox{1.5cm}{\medskip \centering Длина пароля, символы}} & \multicolumn{3}{|c||}{\parbox{6cm}{\centering Энтропия паролей пользователей по критериям NIST}} & \multirow{2}{*}{\parbox{3cm}{\centering Энтропия случайных равновероятных паролей}} \\
        \cline{2-4}
        & \parbox{1.5cm}{\centering Без проверок} & \parbox{2cm}{\centering Словарная проверка} & \parbox{3cm}{\centering Словарная и композиционная проверка} & \\
        \hline
        4  & 10 & 14 & 16 & 26.3 \\
        6  & 14 & 20 & 23 & 39.5 \\
        8  & 18 & 24 & 30 & 52.7 \\
        10 & 21 & 26 & 32 & 65.9 \\
        12 & 24 & 28 & 34 & 79.0 \\
        16 & 30 & 32 & 38 & 105.4 \\
        20 & 36 & 36 & 42 & 131.7 \\
        24 & 40 & 40 & 46 & 158.0 \\
        30 & 46 & 46 & 52 & 197.2 \\
        40 & 56 & 56 & 62 & 263.4 \\
        \hline
    \end{tabular} }
\end{table}

В таблице~\ref{tab:password-entropy} приведена оценка NIST на величину энтропии пользовательских паролей в зависимости от их длины, и приведено сравнение с энтропией случайных паролей с равномерным распределением символов из набора в $D=94$ символов клавиатуры. Вероятное число попыток для подбора пароля составляет $O(2^H)$. Из таблицы видно, что по критериям NIST энтропия реальных паролей в 2--4 раза меньше энтропии случайных паролей с равномерным распределением символов.

\example
Оценим общее количество существующих паролей. Население Земли -- 7 млрд. Предположим, что всё население использует компьютеры и Интернет, и у каждого человека по 10 паролей. Общее количество существующих паролей -- $7 \cdot 10^{10} \approx 2^{36}$.

Имея доступ к наиболее массовым интернет-сервисам с количеством пользователей десятки и сотни миллионов, в которых пароли часто хранятся в открытом виде из-за необходимости обновления ПО и, в частности, выполнения аутентификации, мы:
\begin{enumerate}
	\item имеем базу паролей, покрывающую существенную часть пользователей; 
	\item можем статистически построить правила генерирования паролей.
\end{enumerate}

Даже если пароль хранится в защищённом виде, то при вводе пароль, как правило, в открытом виде пересылается по Интернету, и все преобразования пароля для аутентификации осуществляет интернет-сервис, а не веб-браузер. Следовательно, интернет-сервис имеет доступ к исходному паролю.
\exampleend

В 2002 г. был подобран ключ для 64-битного блочного шифра RC5 сетью персональных компьютеров \texttt{distributed.net}, выполнявших вычисления в фоновом режиме. Суммарное время вычислений всех компьютеров -- 1757 дней, было проверено 83\% пространства всех ключей. Это означает, что пароли с оценочной энтропией менее 64 бит, то есть \emph{все пароли} до 40 символов по критериям NIST, могут быть подобраны в настоящее время. Конечно, с оговорками на то, что 1) нет ограничений на количество и скорость попыток аутентификаций, 2) алгоритм генерации вероятных паролей эффективен.

Строго говоря, использование даже 40-символьного пароля для аутентификации или в качестве ключа блочного шифрования является небезопасным.


\subsubsection{Число паролей}

Приведём различные оценки числа паролей, создаваемых людьми. Чаще всего такие пароли основаны на словах или закономерностях естественного языка. В английском языке всего около $1\ 000\ 000 \approx 2^{20}$ слов, включая термины.

%http://www.springerlink.com/content/bh216312577r6w64/fulltext.pdf
%http://www.antimoon.com/forum/2004/4797.htm

Используемые слоги английского языка имеют вид V, CV, VC, CVV, VCC, CVC, CCV, CVCC, CVCCC, CCVCC, CCCVCC, где C -- согласная (consonant), V -- гласная (vowel). 70\% слогов имеют структуру VC или CVC. Общее число слогов $S = 8000 - 12000$. Средняя длина слога -- 3 буквы.

Предполагая равновероятное распределение всех слогов английского языка, для числа паролей из $r$ слогов получим верхнюю оценку
    \[ N_1 = S^r = 2^{13 r} \approx 2^{4.3 L_1}. \]
Средняя длина паролей составит
    \[ L_1 \approx 3 r. \]

Теперь предположим, что пароли могут состоять только из 2--3 буквенных слогов вида CV, VC, CVV, VCC, CVC, CCV с равновероятным распределением символов. Подсчитаем число паролей $N_2$, которые могут быть построены из $r$ таких слогов. В английском алфавите число гласных букв $n_v = 10, согласных n_c = 16, n = n_v + n_c = 26$. Верхняя оценка числа $r$-слоговых паролей:
    \[ N_2 = (n_c n_v + n_v n_c + n_c n_v n_v + n_v n_c n_c + n_c n_v n_c + n_c n_c n_v)^r \approx \]
        \[ \approx \left( n_c n_v(3 n_c + n_v) \right)^r, \]
    \[ N_2 \approx \left( \frac{n^3}{2} \right)^r \approx 2^{13 r} \approx 2^{4.3 L_2}. \]
Средняя длина паролей:
    \[ L_2 = \frac{n_c n_v(2 + 2 + 3 n_v + 3 n_c + 3 n_c + 3 n_c)}{n_c n_v (1 + 1 + n_v + n_c + n_c + n_c)} \cdot r \approx 3 r. \]

Как видно, в обоих предположениях получились одинаковые оценки для числа и длины паролей.

Подсчитаем верхние оценки числа паролей из $L$ символов, предполагая равномерное распределение символов из алфавита мощностью $D$ символов: a) $D_1 = 26$ строчных букв, б) все $D_2 = 94$ печатных символа клавиатуры (латиница и небуквенные символы):
    \[ N_3 = D_1^L \approx 2^{4.7 L}, \]
    \[ N_4 = D_2^L \approx 2^{6.6 L}. \]

\begin{table}[!ht]
    \caption{Различные верхние оценки числа паролей\label{tab:password-number}}
    \resizebox{\textwidth}{!}{ \begin{tabular}{|c||c|c|c|}
        \hline
        \multirow{2}{*}{\parbox{1.5cm}{\medskip\medspace \centering Длина пароля}} & \multicolumn{3}{|c|}{Число паролей} \\
        \cline{2-4}
            & \parbox{3.5cm}{\medspace \centering На основе слоговой композиции} &
            \parbox{3cm}{\medspace\centering Алфавит $D=26$ символов} &
            \parbox{3cm}{\medspace \centering Алфавит $D=94$ символа} \\
        \hline
        \rule{0pt}{2.5ex}$6$  & $2^{26}$ & $2^{28}$ & $2^{39}$ \\
        9  & $2^{39}$ & $2^{42}$ & $2^{59}$ \\
        12 & $2^{52}$ & $2^{56}$ & $2^{79}$ \\
        15 & $2^{65}$ & $2^{71}$ & $2^{98}$ \\
        \hline
        \rule{0pt}{2.5ex} 21 & $2^{91}$ & $2^{99}$ & $2^{137}$ \\
        \hline
        \rule{0pt}{2.5ex} 39 & $2^{169}$ & $2^{183}$ & $2^{256}$ \\
        \hline
    \end{tabular} }
\end{table}

Из таблицы~\ref{tab:password-number} видно, что при доступном объёме вычислений в $2^{60}$\,--\,$2^{70}$ операций, пароли вплоть до 15 символов, построенные на словах, слогах, изменениях слов, вставках цифр, небольшом изменении регистров и других простейших модификациях, в настоящее время могут быть найдены полным перебором как на вычислительном кластере, так и на персональном компьютере.

Для достижения криптостойкости паролей, сравнимой со 128- или 256-битовым секретным ключом, требуется выбирать пароль из 20 и 40 символов соответственно, что, как правило, не реализуется из-за сложности запоминания и ввода без ошибок.


%Подсчитаем число паролей $N_1$, которые могут могут построены из $r$ ~ 2-3 буквенных слогов: $cv, vc, ccv, cvc, vcc$, где $c$ -- согласная, $v$ -- гласная. В английском алфавите $n_v = 10, n_c = 16, n = n_v + n_c = 26$. Число паролей
%    \[ N_1 = \left( n_v n_c (1 + 1 + n_c + n_c + n_c) \right)^r \approx 3^r n_v^r n_c^{2r}. \]
%Средняя длина паролей
%    \[ L = r \left( \frac{2 + 2 + 3 n_c + 3 n_c + 3 n_c}{1 + 1 + n_c + n_c + n_c} \right) \approx 3r. \]
%
%%Учтем, что $b \leq r$ символов могут быть заглавными: $N_1 \rightarrow N_2 < N_1 \binom{L}{b} \left( \frac{n}{n_v} \right)^b$. Вставим $d$ цифр в случайные места: $N_2 \rightarrow N_3 = N_2 (10 (1 + L))^d \approx N_2 (10 L)^d$.
%%
%%Общее число паролей
%%    \[ N = N_3 = 3^r 10^r 16^{2r} \binom{3r}{b} 2.6^b \left(10 \cdot 3 r \right)^d. \]
%%
%%\begin{table}[!ht]
%%    \centering
%%    \small
%%    \begin{tabular}{|c|c|c|c|c||cr|}
%%        \hline
%%        \parbox{1.3cm}{Слогов, $r$} & \parbox{1.8cm}{Заглавных букв, $b$} & \parbox{1.5cm}{Вставок цифр, $d$} & \parbox{2.8cm}{Средняя длина пароля, $L+d$} & \parbox{3cm}{Верхняя оценка числа паролей $N$} & \multicolumn{2}{|c|}{\parbox{3.2cm}{Число всех паролей}} \\
%%        \hline
%%        $2$ & $0$ & $0$ & $6$ & $2^{26}$ & $2^{36}$ & a-z \\
%%        $2$ & $2$ & $0$ & $6$ & $2^{32}$ & $2^{48}$ & A-Z, a-z \\
%%        $2$ & $2$ & $2$ & $8$ & $2^{45}$ & $2^{48}$ & A-Z, a-z, 0-9 \\
%%        \hline
%%        $3$ & $0$ & $0$ & $9$ & $2^{39}$ & $2^{54}$ & a-z \\
%%        $3$ & $3$ & $0$ & $9$ & $2^{49}$ & $2^{54}$ & A-Z, a-z \\
%%        $3$ & $3$ & $2$ & $11$ & $2^{63}$ & $2^{65}$ & A-Z, a-z, 0-9 \\
%%        \hline
%%        $4$ & $0$ & $0$ & $12$ & $2^{52}$ & $2^{93}$ & a-z \\
%%        $4$ & $3$ & $0$ & $12$ & $2^{64}$ & $2^{186}$ & A-Z, a-z \\
%%        $4$ & $3$ & $2$ & $14$ & $2^{78}$ & $2^{222}$ & A-Z, a-z, 0-9 \\
%%        \hline
%%    \end{tabular}
%%    \caption{Сравнение верхней оценки числа паролей, построенных на слогах, со всем доступным множеством паролей.}
%%    \label{tab:password-number}
%%\end{table}
%
%Учтем, что $b$ символов в пароле могут быть взяты не из 26-символьного алфавита строчных букв, а из всего алфавита в $D=94$ печатных символа клавиатуры (латиница и небуквенные символы):
%\[
%    \begin{array}{ll}
%    b=1 & N_1 \rightarrow N_2 = \frac{n_v}{n_v+n_c} 3^r n_v^{r-1} n_c^{2r} \cdot L. \]
%
%    \[ N_1 \rightarrow N_2 < N_1 \binom{L}{b} \left( \frac{D}{n_v} \right)^b. \]
%
%
%
%Общее число паролей
%    \[ N < 3^r n_v^r n_c^{2r} \binom{L}{b} \left( \frac{D}{n_v} \right)^b = 3^r 10^r 16^{2r} \binom{3r}{b} \left( \frac{94}{10} \right)^b. \]
%
%\begin{table}[!ht]
%    \centering
%    \small
%    \begin{tabular}{|c|c|c|c||cr|}
%        \hline
%        \parbox{1.5cm}{Слогов, $r$} & \parbox{3cm}{Средняя длина пароля, $L$} & \parbox{3cm}{Символов из всего алфавита, $b$} & \parbox{3cm}{Верхняя оценка числа паролей $N$} & \multicolumn{2}{|c|}{\parbox{3.2cm}{Число всех паролей, $D^L$}} \\
%        \hline
%        \multirow{3}{*}{2} & \multirow{3}{*}{6} & $0$ & $2^{26}$ & $2^{28}$ & a-z \\
%        & & $1$ & $2^{32}$ & $2^{34}$ & A-Z, a-z \\
%        & & $3$ & $2^{40}$ & $2^{39}$ & Весь алфавит \\
%        \hline
%        \multirow{3}{*}{3} & \multirow{3}{*}{9} & $0$ & $2^{39}$ & $2^{42}$ & a-z \\
%        & & $2$ & $2^{50}$ & $2^{51}$ & A-Z, a-z \\
%        & & $4$ & $2^{59}$ & $2^{59}$ & Весь алфавит \\
%        \hline
%        \multirow{3}{*}{4} & \multirow{3}{*}{12} & $0$ & $2^{52}$ & $2^{56}$ & a-z \\
%        & & $3$ & $2^{69}$ & $2^{68}$ & A-Z, a-z \\
%        & & $6$ & $2^{81}$ & $2^{77}$ & Весь алфавит \\
%        \hline
%    \end{tabular}
%    \caption{Сравнение верхней оценки числа паролей, построенных на слогах, со всем доступным множеством паролей в алфавите из $D$ символов.}
%    \label{tab:password-number}
%\end{table}
%
%Из таблицы~\ref{tab:password-number} видно, что при доступном объёме вычислений в $2^{60 \ldots 70}$ операций, пароли вплоть до 12 символов, построенные на словах, слогах, изменениях слов, вставках цифр, небольшого изменения регистров и другой простейшей обфускации, могут быть найдены перебором на кластере (или ПК) в настоящее время.


\subsubsection{Атака для подбора паролей и ключей шифрования}

В схемах аутентификации по паролю иногда используется хэширование и хранение хэша пароля на сервере. В таких случаях применима словарная атака или атака с применением заранее вычисленных таблиц для ускорения поиска.

Для нахождения пароля, прообраза хэш-функции, или для нахождения ключа блочного шифрования по атаке с выбранным шифротекстом (для одного и того же известного открытого текста и соответствующего шифротекста) может быть применён метод перебора с балансом между памятью и временем вычислений. Самый быстрый метод радужных таблиц\index{радужные таблицы} (\langen{rainbow tables}, 2003~г., \cite{Oechslin:2003}) заранее вычисляет следующие цепочки и хранит результат в памяти.

Для нахождения пароля, прообраза хэш-функции $H$, цепочка строится как
    \[ M_0 \xrightarrow{H(M_0)} h_0 \xrightarrow{R_0(h_0)} M_1 \ldots M_t \xrightarrow{H(M_t)} h_t \xrightarrow{R_t(h_t)} M_{t+1}, \]
где $R_i(h)$ -- функция редуцирования, преобразования хэша в пароль для следующего хэширования.

Для нахождения ключа блочного шифрования для одного и того же известного открытого текста $M$ таблица строится как
    \[ K_0 \xrightarrow{E_{K_0}(M)} c_0 \xrightarrow{R_0(c_0)} K_1 \ldots K_t \xrightarrow{E_{K_t}(M)} c_t \xrightarrow{R_t(c_t)} K_{t+1}, \]
где $R_i(c)$ -- функция редуцирования, преобразования шифротекста в новый ключ.

Функция редуцирования $R_i$ зависит от номера итерации, чтобы избежать дублирующихся подцепочек, которые возникают в случае коллизий между значениями в разных цепочках в разных итерациях, если $R$ постоянна. Радужная таблица размера $(m \times 2)$ состоит из строк $(M_{0,j}, M_{t+1,j})$ или $(K_{0,j}, K_{t+1,j})$, вычисленных для разных значений стартовых паролей $M_{0,j}$ или $K_{0,j}$ соответственно.

Опишем атаку на примере нахождения прообраза $\overline{M}$ хэша $\overline{h} = H(\overline{M})$. На первой итерации исходный хэш $\overline{h}$ редуцируется в сообщение $\overline{h} \xrightarrow{R_t(\overline{h})} \overline{M}_{t+1} $ и сравнивается со всеми значениями последнего столбца $M_{t+1,j}$ таблицы. Если нет совпадения, переходим ко второй итерации. Хэш $\overline{h}$ дважды редуцируется в сообщение $\overline{h} \xrightarrow{R_{t-1}(\overline{h})} \overline{M}_t \xrightarrow{H(\overline{M}_t)} \overline{h}_t \xrightarrow{R_t(\overline{h}_t)} \overline{M}_{t+1}$ и сравнивается со всеми значениями последнего столбца $M_{t+1,j}$ таблицы. Если не совпало, то переходим к третьей итерации и~т.\,д. Если для $r$-кратного редуцирования сообщение $\overline{M}_{t+1}$ содержится в таблице во втором столбце, то из совпавшей строки берётся $M_{0,j}$, и вся цепочка пробегается заново для поиска искомого сообщения $\overline{M}: ~ \overline{h} = H(\overline{M})$.

Найдём вероятность нахождения пароля в таблице. Пусть мощность множества всех паролей $N$. Изначально в столбце $M_{0,j}$ содержится $m_0 = m$ различных паролей. Предполагая наличие случайного отображения с пересечениями паролей $M_{0,j} \rightarrow M_{1,j}$, в $M_{1,j}$ будет $m_1$ различных паролей. Согласно задаче о размещении,
\[
    m_{i+1} = N \left( 1 - \left( 1 - \frac{1}{N} \right)^{m_i} \right) \approx N \left( 1 - e^{-\frac{m_i}{N}} \right).
\]
Вероятность нахождения пароля
\[
    P = 1 - \prod \limits_{i=1}^t \left( 1 - \frac{m_i}{N} \right).
\]

Чем больше таблица из $m$ строк, тем больше шансов найти пароль или ключ, выполнив в наихудшем случае   $O \left( m \frac{t(t+1)}{2} \right)$ операций.

Примеры применения атаки на хэш-функциях $\textrm{MD5}$\index{хэш-функция!MD5}, $\textrm{LM} \sim \textrm{DES}_{\textrm{Password}} (\textrm{const})$ приведены в таблице~\ref{tab:rainbow-tables}.

\begin{table}[!ht]
    \centering
    \caption{Атаки на радужных таблицах на \emph{одном} ПК\label{tab:rainbow-tables}}
    \resizebox{\textwidth}{!}{ \begin{tabular}{|c|c|c|c|c|c|c|}
        \hline
        \multirow{2}{*}{\parbox{1.0cm}{\medskip\medskip \centering Длина, биты}} & \multicolumn{3}{|c|}{\parbox{4.3cm}{\medspace\centering Пароль или ключ}} &
            \multicolumn{3}{|c|}{\parbox{4.33cm}{\medspace\centering Радужная таблица}} \\
        \cline{2-7}
        & \parbox{1.0cm}{\centering Длина,\\ симв.} & \parbox{1.7cm}{\centering Множество} & \parbox{1.7cm}{\centering Мощность} &
            \parbox{1cm}{\centering Объём} & \parbox{2.23cm}{\medspace \centering Время вычисления таблиц} & \parbox{1.1cm}{\centering Время поиска} \\
        \hline \hline
        \multicolumn{7}{|c|}{Хэш LM} \\
        \hline
        \rule{0pt}{2.5ex}\multirow{3}{*}{$2 \times 56$} & \multirow{3}{*}{14} &
            A--Z & $2^{33}$ & 610 MB &  & 6 с \\
        & & A--Z, 0-9 & $2^{36}$ & 3 GB &  & 15 с \\
        & & все & $2^{43}$ & 64 GB & \parbox{2.23cm}{несколько лет} & 7 мин \\
        \hline \hline
        \multicolumn{7}{|c|}{Хэш MD5} \\
        \hline
        \rule{0pt}{2.5ex} 128 & 8 & A-Z, 0-9 & $2^{41}$ & 36 GiB & - & 4 мин \\
        \hline
    \end{tabular} }
\end{table}

\section{Аутентификация по паролю}

Из-за малой энтропии пользовательских паролей во всех системах регистрации и аутентификации пользователей применяется специальная политика безопасности. Типичные минимальные требования:
\begin{enumerate}
    \item Длина пароля от 8 символов. Использование разных регистров и небуквенных символов в паролях. Запрет паролей из словаря или часто используемых паролей. Запрет паролей в виде дат, номеров машин и других номеров.
    \item Ограниченное время действия пароля. Обязательная смена пароля по истечении срока действия.
    \item Блокирование возможности аутентификации после нескольких неудачных попыток. Ограниченное число актов аутентификаций в единицу времени. Временная задержка перед выдачей результата аутентификации.
\end{enumerate}

Дополнительные рекомендации (требования) пользователям:
\begin{enumerate}
    \item Не использовать одинаковые или похожие пароли для разных систем, таких как электронная почта, вход в ОС, электронная платёжная система, форумы, социальные сети. Пароль часто передаётся в открытом виде по сети. Пароль доступен администратору системы, возможны утечки конфиденциальной информации с серверов. Поэтому следует стараться выбирать случайные стойкие пароли.
    \item Не записывать пароли. Никому не сообщать пароль, даже администратору. Не передавать пароли по почте, телефону, Интернету и~т.\,д.
    \item Не использовать одну и ту же учётную запись для разных пользователей, даже в виде исключения.
    \item Всегда блокировать компьютер, когда пользователь отлучается от него, даже на короткое время.
\end{enumerate}

\input{os_passwords}

\input{http_auth}

\chapter{Программные уязвимости}

\input{security_models}

\input{os_access_controls}

\section{Виды программных уязвимостей}

\emph{Вирусом} называется самовоспроизводящаяся часть кода (подпрограмма)\index{вирус}, которая встраивается в носители (другие программы) для своего исполнения и распространения. Вирус не может исполняться и передаваться без своего носителя.

\emph{Червём} называется самовоспроизводящаяся отдельная (под)программа\index{червь}, которая может исполняться и распространяться самостоятельно, не используя программу-носитель.

Первой вехой в изучении компьютерных вирусов можно назвать 1949 год, когда Джон фон Нейман прочёл курс лекций в Университете Иллинойса под названием <<Теория самовоспроизводящихся машин>> (изданы в 1966~\cite{Neumann:1966}, переведены на русский язык издательством <<Мир>> в 1971 году~\cite{Neumann:1971}), в котором ввёл понятие самовоспроизводящихся механических машин. Первым сетевым вирусом считается вирус Creeper 1971 г., распространявшийся в сети ARPANET, предшественнике Интернета. Для его уничтожения был создан первый антивирус Reaper, который находил и уничтожал Creeper.

Первый червь для Интернета, червь Морриса 1988 г., уже использовал \emph{смешанные} атаки\index{атака!смешанная} для заражения UNIX машин~\cite{EichinRochlis:1988, Spafford:1989}. Сначала программа получала доступ к удалённому запуску команд, эксплуатируя уязвимости в сервисах \texttt{sendmail}, \texttt{finger} (с использованием атаки на переполнение буфера) или \texttt{rsh}. Далее, с помощью механизма подбора паролей червь получал доступ к локальным аккаунтам пользователей:
\begin{itemize}
    \item получение доступа к учётным записям с очевидными паролями:
		\begin{itemize}
			\item без пароля вообще;
			\item имя аккаунта в качестве пароля;
			\item имя аккаунта в качестве пароля, повторённое дважды;
			\item использование <<ника>> (\langen{nickname});
			\item фамилия (\langen{last name, family name});
			\item фамилия, записанная задом наперёд;
		\end{itemize}
		\item перебор паролей на основе встроенного словаря из 432 слов;
		\item перебор паролей на основе системного словаря \texttt{/usr/dict/words}.
\end{itemize}

\emph{Программной уязвимостью}\index{программная уязвимость} называется свойство программы, позволяющее нарушить её работу. Программные уязвимости могут приводить к отказу в обслуживании (Denial of Service, DoS-атака)\index{атака!отказ в обслуживании}, утечке и изменению данных, появлению и распространению вирусов и червей.

Одной из распространённых атак для заражения персональных компьютеров является переполнение буфера в стеке. В интернет-сервисах наиболее распространённой программной уязвимостью в настоящее время является межсайтовый скриптинг (Cross-Site Scripting, XSS-атака)\index{атака!XSS}.

Наиболее распространённые программные уязвимости можно разделить на классы:
\begin{enumerate}
    \item Переполнение буфера -- копирование в буфер данных большего размера, чем длина выделенного буфера. Буфером может быть контейнер текстовой строки, массив, динамически выделяемая память и~т.\,д. Переполнение становится возможным, вследствие либо отсутствия контроля над длиной копируемых данных, либо из-за ошибок в коде. Типичная ошибка -- разница в 1 байт между размерами буфера и данных при сравнении.
    \item Некорректная обработка (парсинг) данных, введённых пользователем, является причиной большинства программных уязвимостей в веб-приложениях. Под обработкой понимаются:
        \begin{enumerate}
            \item проверка на допустимые значения и тип (числовые поля не должны содержать строки и~т.\,д.);
            \item фильтрация и экранирование специальных символов, имеющих значения в скриптовых языках или применяющихся для декодирования из одной текстовой кодировки в другую. Примеры символов: \texttt{\textbackslash}, \texttt{\%}, \texttt{<}, \texttt{>}, \texttt{"}, \texttt{'};
            \item фильтрация ключевых слов языков разметки и скриптов. Примеры: \texttt{script}, \texttt{JavaScript};
            \item декодирование различными кодировками при парсинге. Распространённый способ обхода системы контроля парсинга данных состоит в однократном или множественном последовательном кодировании текстовых данных в шестнадцатеричные кодировки \texttt{\%NN} ASCII и UTF-8. Например, браузер или веб-приложения производят $n$-кратные последовательные декодирования, в то время как система контроля делает $k$-кратное декодирование, $0 \leq k < n$, и, следовательно, пропускает закодированные запрещённые символы и слова.
        \end{enumerate}
    \item Некорректное использование синтаксиса функций. Например, \texttt{printf(s)} может привести к уязвимости записи в указанный адрес памяти. Если злоумышленник вместо обычной текстовой строки введёт в качестве \texttt{s = "текст некоторой длины\%n"}, то функция \texttt{printf()}, ожидающая первым аргументом строку формата \texttt{printf(fmt, \dots)}, обнаружив \texttt{\%n}, возьмёт значения из ячеек памяти, следующих перед текстовой строкой (устройство стека функции описано далее), и запишет в адрес памяти, равный считанному значению, количество выведенных символов на печать функцией \texttt{printf()}.
\end{enumerate}


\input{stack_overflow}

\input{xss}

\input{sql-injections}

%\chapter{Послесловие}
%Это должно быть что-то в виде заключения, объяснения, почему именно эти темы выбраны, насколько актуален материал с теоретической и практической точки зрения.


\appendix
\renewcommand{\thechapter}{\Asbuk{chapter}} % использование русских букв для нумерации приложений

\chapter{Математическое приложение}\label{chap:discrete-math}

\section{Общие определения}

Выражением $\Mod n$ обозначается вычисление остатка от деления произвольного целого числа на целое число $n$. В полиномиальной арифметике эта операция означает вычисление остатка от деления многочленов.
%далее будем обозначать целые числа или операции с целыми числами, взятыми \emph{по модулю}\index{модуль} целого числа $n$ (остаток от целочисленного деления). Примеры выражений:
    \[ a\mod n, \]
    \[ (a + b)\cdot c\mod n. \]
Равенство
    \[ a = b \mod n \]
означает, что выражения $a$ и $b$ равны (говорят также <<сравнимы>>, <<эквивалентны>>) по модулю $n$.

Множество
    \[ \{ 0, 1, 2, 3, \dots, n-1 \mod n\} \]
состоит из $n$ элементов, где каждый элемент $i$ представляет все целые числа, сравнимые с $i$ по модулю $n$.

Наибольший общий делитель (НОД) двух чисел $a,b$ обозначается $\gcd(a,b)$ (\textit{greatest common divisor}).

Два числа $a,b$ называются взаимно простыми, если они не имеют общих делителей, кроме 1, то есть $\gcd(a,b) = 1$.

Выражение $a \mid b$ означает, что $a$ делит $b$.

\input{birthdays_paradox}

\input{groups}

\input{aes_math}

\input{modular_ariphmetics}

\input{pseudo-primes}

\input{groups_of_ec_points_over_finite_fields}

\section[Полиномиальные и экспоненциальные алгоритмы]{Полиномиальные и \\ экспоненциальные алгоритмы}

Данный раздел поясняет обоснованность стойкости криптосистем с открытым ключом и имеет лишь косвенное отношение к дискретной математике.

Машина Тьюринга (МТ) (модель, представляющая любой вычислительный алгоритм) состоит из следующих частей:
\begin{itemize}
    \item неограниченная лента, разделённая на клетки; в каждой клетке содержится символ из конечного алфавита, содержащего пустой символ blank; если символ ранее не был записан на ленту, то он считается blank;
    \item печатающая головка, которая может считать, записать символ $a_i$ и передвинуть ленту на 1 клетку влево или вправо $d_k$;
    \item конечная таблица действий
    \[ (q_i, a_j) \rightarrow (q_{i1}, a_{j1}, d_k), \]
где $q$ -- состояние машины.
\end{itemize}

Если таблица переходов однозначна, то машина Тьюринга\index{машина Тьюринга} называется детерминированной. \emph{Детерминированная} машина Тьюринга может \emph{имитировать} любую существующую детерминированную ЭВМ. Если таблица переходов неоднозначна, то есть $(q_i, a_j)$ может переходить по нескольким правилам, то машина \emph{недетерминированная}. \emph{Квантовый компьютер} является примером недетерминированной машины Тьюринга.

Класс задач $\set{P}$ -- задачи, которые могут быть решены за \emph{полиномиальное} время\index{задача!полиномиальная} на \emph{детерминированной} машине Тьюринга. Пример полиномиальной сложности (количество битовых операций)
    \[ O(k^{\textrm{const}}), \]
где $k$ -- длина входных параметров алгоритма. Операция возведения в степень в модульной арифметике $a^b \mod n$ имеет кубическую сложность $O(k^3)$, где $k$ -- двоичная длина чисел $a,b,n$.

Класс задач $\set{NP}$ -- обобщение класса $\set{P} \subseteq \set{NP}$, включает задачи, которые могут быть решены за \emph{полиномиальное} время на \emph{недетерминированной} машине Тьюринга. Пример сложности задач из $\set{NP}$ -- экспоненциальная сложность\index{задача!экспоненциальная}
    \[ O(\textrm{const}^k). \]
Описанный в разделе криптостойкости системы Эль-Гамаля\index{криптосистема!Эль-Гамаля} алгоритм Гельфонда решения задачи дискретного логарифмирования (нахождения $x$ для заданных основания $g$, модуля $p$ и $a = g^x \mod p$) имеет сложность $O(e^{k/2})$, где $k$ -- двоичная длина чисел.

В криптографии полиномиальные $\set{P}$ алгоритмы считаются \emph{лёгкими и вычислимыми} на ЭВМ, которые являются детерминированными машинами Тьюринга. Неполиномиальные (экспоненциальные) $\set{NP}$ алгоритмы считаются \emph{трудными и невычислимыми} на ЭВМ, так как из-за экспоненциального роста сложности всегда можно выбрать такой параметр $k$, что время вычисления станет сравнимым с возрастом Вселенной.


Класс $\set{NP}$-полных задач -- подмножество задач из $\set{NP}$, для которых не известен полиномиальный алгоритм для детерминированной машины Тьюринга, и все задачи могут быть сведены друг к другу за полиномиальное время на \emph{детерминированной} машине Тьюринга. Например, задача об укладке рюкзака является $\set{NP}$-полной.

Стойкость криптосистем с \emph{открытым} ключом, как правило, основана на $\set{NP}$ или $\set{NP}$-полных задачах:
\begin{enumerate}
    \item RSA\index{криптосистема!RSA} -- $\set{NP}$-задача факторизации (строго говоря, основана на трудности извлечения корня степени $e$ по модулю $n$).
    \item Криптосистемы типа Эль-Гамаля\index{криптосистема!Эль-Гамаля} -- $\set{NP}$-задача дискретного логарифмирования.
\end{enumerate}

\emph{Нерешённой} проблемой является доказательство неравенства
    \[ \set{P} \neq \set{NP}. \]
Именно на гипотезе о том, что для некоторых задач не существует полиномиальных алгоритмов, и основана стойкость криптосистем с открытым ключом.

\input{coincide-index_method}

\input{tasks}

\printindex

\chapter*{Литература}
\addcontentsline{toc}{chapter}{Литература}
\begingroup
\renewcommand{\chapter}[2]{}%
%\bibliographystyle{ugost2008s}
%\bibliography{bibliography}
\printbibliography
\endgroup

\end{document}
